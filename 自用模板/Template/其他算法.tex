\section{其他算法}

\subsection{二分 \& 三分}
\begin{lstlisting}
const double eps = 1e-10;
double l, r, a[MAXN];
int n;
double F(double x) {}

void Solve() {
	// 单调递增序列a中查找 >=x 的数中最小的一个 
	while(l < r) {
		int mid = l + r >> 1;
		if(a[mid] >= x) r = mid;
		else l = mid + 1;
	}
	
	// 单调递增序列a中查找 <=x 的数中最大的一个
	while(l < r) {
		int mid = l + r + 1 >> 1;
		if(a[mid] <= x) l = mid;
		else r = mid - 1;
	}
}

void Solve() {	
	// 浮点数三分 求单峰函数的极大值
	while(r - l > eps) {
		double lmid = 1.0 * l + 1.0 * (r - l) / 3.0;
		double rmid = 1.0 * r - 1.0 * (r - l) / 3.0;
		if(F(lmid) > F(rmid)) r = rmid;
		else l = lmid;
	}
	
	// 整数三分 注意特判边界等各种情况 较毒瘤
	while(l < r) {
		int lmid = l + (r - l) / 3;
		int rmid = r - (r - l) / 3;
		if(F(lmid) > F(rmid)) r = rmid;
		else l = lmid;		
	}
}
\end{lstlisting}

\subsection{前缀和 \& 差分}
\begin{lstlisting}
// 二维前缀和 & 二维差分
sum[i][j] = a[i][j] + sum[i - 1][j] + sum[i][j - 1] - sum[i - 1][j - 1];
d[i][j] = a[i][j] - a[i - 1][j] - a[i][j - 1] + a[i - 1][j - 1];

int QuerySum() {
	// 以(x1, y1)为左上角,(x2, y2)为右下角的子矩阵的和为:
	return sum[x2][y2] - sum[x1 - 1][y2] - sum[x2][y1 - 1] + sum[x1 - 1][y1 - 1];
}
void ModifyD() {
	// 给以(x1, y1)为左上角,(x2, y2)为右下角的子矩阵中的所有元素加上c:
	d[x1][y1] += c, d[x2 + 1][y1] -= c, d[x1][y2 + 1] -= c, d[x2 + 1][y2 + 1] += c;
}

/*
树上差分
注意对差分数组进行前缀和操作,得到原数组时顺序是 自叶子节点开始 到 根结束 !
*/
void SolveNode(int u, int v, int t) { //u--v之间的点全部加上c
	val[u] += t, val[v] += t;
	int lca = LCA(u, v);
	val[lca] -= t, val[fa[lca]] -= t;
}

void SolveEdge(int u, int v, int t) { //u--v之间的边全部加上c
	val[u] += t, val[v] += t;
	val[LCA(u, v)] -= 2 * t;
}
\end{lstlisting}

\subsection{离散化}
\begin{lstlisting}
// a[i] 为初始数组,下标范围为 [1, n]
// len 为离散化后数组的有效长度
std::sort(a + 1, a + 1 + n);
len = std::unique(a + 1, a + n + 1) - a - 1;
// 离散化整个数组的同时求出离散化后本质不同数的个数。
std::lower_bound(a + 1, a + len + 1, x) - a;  // 查询 x 离散化后对应的编号
\end{lstlisting}

\subsection{排序}
\begin{lstlisting}
void QuickSort(int l, int r) { //快排
	int mid = a[(l + r) / 2];
	int i = l, j = r;
	do {
		while(a[i] < mid) i++;
		while(a[j] > mid) j--;
		if(i <= j) swap(a[i], a[j]), i++, j--;
	} while(i <= j);
	if(j > l) QuickSort(l, j);
	if(i < r) QuickSort(i, r);
}

void MergeSort(int l, int r) { //归并排序
	if (l >= r) return;
	int mid = (l + r) >> 1;
	MergeSort(l, mid);
	MergeSort(mid + 1, r);
	int k = 0, i = l, j = mid + 1;
	while(i <= mid && j <= r) {
		if(a[i] <= a[j]) tmp[k++] = a[i++];
		else tmp[k++] = a[j++];
	}
	while(i <= mid) tmp[k++] = a[i++];
	while(j <= r) tmp[k++] = a[j++];
	for(i = l, j = 0; i <= r; i++, j++) a[i] = tmp[j];
}
\end{lstlisting}

\subsection{高精度运算}
\begin{lstlisting}
// C = A + B, A >= 0, B >= 0
vector<int> add(vector<int> &A, vector<int> &B) {
	if (A.size() < B.size()) return add(B, A);
	vector<int> C;
	int t = 0;
	for (int i = 0; i < A.size(); i ++ ) {
		t += A[i];
		if (i < B.size()) t += B[i];
		C.push_back(t % 10);
		t /= 10;
	}
	if (t) C.push_back(t);
	return C;
}

// C = A - B, ТњзуA >= B, A >= 0, B >= 0
vector<int> sub(vector<int> &A, vector<int> &B) {
	vector<int> C;
	for (int i = 0, t = 0; i < A.size(); i ++ ) {
		t = A[i] - t;
		if (i < B.size()) t -= B[i];
		C.push_back((t + 10) % 10);
		if (t < 0) t = 1;
		else t = 0;
	}
	while (C.size() > 1 && C.back() == 0) C.pop_back();
	return C;
}

// C = A * b, A >= 0, b >= 0
vector<int> mul(vector<int> &A, int b) {
	vector<int> C;
	int t = 0;
	for (int i = 0; i < A.size() || t; i ++ ) {
		if (i < A.size()) t += A[i] * b;
		C.push_back(t % 10);
		t /= 10;
	}
	while (C.size() > 1 && C.back() == 0) C.pop_back();
	return C;
}

// A / b = C ... r, A >= 0, b > 0
vector<int> div(vector<int> &A, int b, int &r) {
	vector<int> C;
	r = 0;
	for (int i = A.size() - 1; i >= 0; i -- ) {
		r = r * 10 + A[i];
		C.push_back(r / b);
		r %= b;
	}
	reverse(C.begin(), C.end());
	while (C.size() > 1 && C.back() == 0) C.pop_back();
	return C;
}
\end{lstlisting}

\subsection{CDQ分治}
\begin{lstlisting}
#pragma GCC optimize(2)
#define IOS ios::sync_with_stdio(false); cin.tie(0);
#include<bits/stdc++.h>
#define int long long
using namespace std;

const int MAXN = (int)1e5 + 5;

struct Node {
	int val, del, ans;
}a[MAXN];

int n, m, ans, c[MAXN], pos[MAXN];

bool cmp1(const Node &a, const Node &b) { return a.val < b.val; }
bool cmp2(const Node &a, const Node &b) { return a.del < b.del; }

void Modify(int x, int k) {
	while(x <= n) {
		c[x] += k;
		x += (x & (-x));
	}
}

int Query(int x) {
	int res = 0;
	while(x > 0) {
		res += c[x];
		x -= (x & (-x));
	}
	return res;
}

void Solve(int l, int r) {
	if(l == r) return ;
	int mid = l + r >> 1;
	Solve(l, mid), Solve(mid + 1, r);
	
	int i = l, j = mid + 1;
	while(i <= mid) {
		while(a[i].val > a[j].val && j <= r) Modify(a[j].del, 1), ++j;
		a[i].ans += Query(m + 1) - Query(a[i].del), ++i;
	}
	i = l, j = mid + 1;
	while(i <= mid) {
		while(a[i].val > a[j].val && j <= r) Modify(a[j].del, -1), ++j;
		++i;
	}
	
	i = mid, j = r;
	while(j > mid) {
		while(a[j].val < a[i].val && i >= l) Modify(a[i].del, 1), --i;
		a[j].ans += Query(m + 1) - Query(a[j].del), --j;
	}
	i = mid, j = r;
	while(j > mid) {
		while(a[j].val < a[i].val && i >= l) Modify(a[i].del, -1), --i;
		--j;
	}
	sort(a + l, a + r + 1, cmp1);
}

signed main()
{
	IOS
	cin >> n >> m;
	for(int i = 1; i <= n; i++) cin >> a[i].val, pos[ a[i].val ] = i;
	for(int i = 1; i <= m; i++) {
		int cur; cin >> cur;
		a[ pos[cur] ].del = i;
	}
	for(int i = 1; i <= n; i++) if(!a[i].del) a[i].del = m + 1;
	for(int i = 1; i <= n; i++) {
		ans += Query(n) - Query(a[i].val);
		Modify(a[i].val, 1);
	}
	for(int i = 1; i <= n; i++) Modify(a[i].val, -1);
	Solve(1, n);
	sort(a + 1, a + n + 1, cmp2);
	for(int i = 1; i <= m; i++) {
		cout << ans << "\n";
		ans -= a[i].ans;
	}
	return 0;
}
\end{lstlisting}