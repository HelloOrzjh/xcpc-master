\section{数据结构}

\subsection{并查集}
\begin{lstlisting}
struct DSU {    // 按秩合并 可撤销
	struct Node {
	    int u, v, val;
    };
    
    int f[MAXN], dep[MAXN], top = 0;
	Node s[MAXN];
	
	void Init(int n) { for(int i = 1; i <= n; i++) f[i] = i, dep[i] = 0; }
	
	int Find(int u) { return u == f[u] ? u : Find(f[u]); }
	
	void Union(int u, int v) {  // 按秩合并
		u = Find(u), v = Find(v);
		if(u == v) return ;
		if(dep[u] > dep[v]) swap(u, v);
		int val = (dep[u] == dep[v]);
		s[++top] = (Node){u, v, val};
		f[u] = v, dep[v] += val;
	}
	
	bool Same(int u, int v) { return Find(u) == Find(v); }
	
	void Undo(int cur) {
		while(top > cur) {
			int u = s[top].u, v = s[top].v, val = s[top].val;	// dep[u] <= dep[v]
			f[u] = u, dep[v] -= val, top--;
		}
	}
}dsu;
\end{lstlisting}

\subsection{可持久化并查集}
\begin{lstlisting}
//#pragma GCC optimize(2)
#include<bits/stdc++.h>
using namespace std;
const int MAXN = 200005;
const int MOD = 1e9 + 7;

struct Node {
	int fa, dep, ls, rs;
}tree[MAXN * 40];

int cnt, n, m, root[MAXN];

int Build(int l, int r) {
	int dir = ++cnt;
	if(l == r) { tree[dir].fa = l; return dir; }
	int mid = l + r >> 1; 
	tree[dir].ls = Build(l, mid);
	tree[dir].rs = Build(mid + 1, r);
	return dir;
}

int ModifyDep(int rt, int l, int r, int loc, int val) {
	int dir = ++cnt;
	tree[dir] = tree[rt];
	if(l == r) { tree[dir].dep += val; return dir; }
	int mid = (l + r) >> 1;
	if(loc <= mid) tree[dir].ls = ModifyDep(tree[dir].ls, l, mid, loc, val);
	if(loc > mid) tree[dir].rs = ModifyDep(tree[dir].rs, mid + 1, r, loc, val); 
	return dir;
}

int ModifyFa(int rt, int l, int r, int loc, int val) {
	int dir = ++cnt;
	tree[dir] = tree[rt];
	if(l == r) { tree[dir].fa = val; return dir; }
	int mid = (l + r) >> 1;
	if(loc <= mid) tree[dir].ls = ModifyFa(tree[dir].ls, l, mid, loc, val);
	if(loc > mid) tree[dir].rs = ModifyFa(tree[dir].rs, mid + 1, r, loc, val); 
	return dir;
}

int QueryNode(int rt, int l, int r, int loc) {
	if(l == r) return rt;
	int mid = (l + r) >> 1;
	if(loc <= mid) return QueryNode(tree[rt].ls, l, mid, loc);
	if(loc > mid) return QueryNode(tree[rt].rs, mid + 1, r, loc);
}

int Find(int rt, int u) {	// 返回值为祖先节点在主席树上对应节点编号
	int cur = QueryNode(rt, 1, n, u);
	if(u == tree[cur].fa) return cur;
	return Find(rt, tree[cur].fa); 
}

void Union(int p, int u, int v) {	// 按秩合并
	u = Find(root[p], u), v = Find(root[p], v);
	if(u == v) return ;
	if(tree[u].dep > tree[v].dep) swap(u, v);
	root[p] = ModifyFa(root[p], 1, n, tree[u].fa, tree[v].fa);
	if(tree[u].dep == tree[v].dep) root[p] = ModifyDep(root[p], 1, n, tree[v].fa, 1);
}

signed main()
{
	ios::sync_with_stdio(false); cin.tie(0); cout.tie(0);
	cin >> n >> m; root[0] = Build(1, n);
	for(int i = 1; i <= m; i++) {
		root[i] = root[i - 1];
		int op, a, b, k; cin >> op;
		if(op == 1) cin >> a >> b, Union(i, a, b);
		else if(op == 2) cin >> k, root[i] = root[k];
		else cin >> a >> b, cout << (tree[ Find(root[i], a) ].fa == tree[ Find(root[i], b) ].fa) << "\n";
	}
	return 0;
}
\end{lstlisting}


\subsection{K-D Tree}
\begin{lstlisting}
//#pragma GCC optimize(2)
#include<bits/stdc++.h>
#define x1 x123456789
#define y1 y123456789
using namespace std;

const int MAXN = 5e5 + 5;
const double alpha = 0.725;

struct node {
	int x, y, v;
}a[MAXN];

int n, cnt;
int rt, L[MAXN], R[MAXN], D[MAXN], U[MAXN], d[MAXN], sze[MAXN], sum[MAXN], ls[MAXN], rs[MAXN];
int g[MAXN], t;

bool cmp1(int x, int y) { return a[x].x < a[y].x; }
bool cmp2(int x, int y) { return a[x].y < a[y].y; }

void maintain(int p) {
	sze[p] = sze[ ls[p] ] + sze[ rs[p] ] + 1;
	sum[p] = sum[ ls[p] ] + sum[ rs[p] ] + a[p].v;
	L[p] = R[p] = a[p].x;
	D[p] = U[p] = a[p].y;
	if(ls[p]) {
		L[p] = min(L[p], L[ ls[p] ]), R[p] = max(R[p], R[ ls[p] ]);
		D[p] = min(D[p], D[ ls[p] ]), U[p] = max(U[p], U[ ls[p] ]);
	}
	if(rs[p]) {
		L[p] = min(L[p], L[ rs[p] ]), R[p] = max(R[p], R[ rs[p] ]);
		D[p] = min(D[p], D[ rs[p] ]), U[p] = max(U[p], U[ rs[p] ]);
	}
}

int build(int l, int r) {
	if(l > r) return 0;
	int mid = l + r >> 1;
	double ave1 = 0, ave2 = 0, var1 = 0, var2 = 0;
	for(int i = l; i <= r; i++) ave1 += a[ g[i] ].x, ave2 += a[ g[i] ].y;
	ave1 /= (r - l + 1), ave2 /= (r - l + 1);
	for(int i = l; i <= r; i++) {
		var1 += (a[ g[i] ].x - ave1) * (a[ g[i] ].x - ave1);
		var2 += (a[ g[i] ].y - ave2) * (a[ g[i] ].y - ave2);
	}
	if(var1 > var2) {
		nth_element(g + l, g + mid, g + r + 1, cmp1);
		d[ g[mid] ] = 1;
	} else {
		nth_element(g + l, g + mid, g + r + 1, cmp2);
		d[ g[mid] ] = 2;
	}
	ls[ g[mid] ] = build(l, mid - 1);
	rs[ g[mid] ] = build(mid + 1, r);
	maintain(g[mid]);
	return g[mid];
}

void print(int p) {
	if(!p) return ;
	print(ls[p]);
	g[++t] = p;
	print(rs[p]);
}

void rebuild(int &p) {
	t = 0;
	print(p);
	p = build(1, t);
}

bool bad(int p) {
	return alpha * sze[p] <= (double) max(sze[ ls[p] ], sze[ rs[p] ]);
}

void insert(int &p, int k) {
	if(!p) {
		p = k;
		maintain(p);
		return ;
	}
	if(d[p] == 1) {
		if(a[k].x <= a[p].x) insert(ls[p], k);
		else insert(rs[p], k);
	} else {
		if(a[k].y <= a[p].y) insert(ls[p], k);
		else insert(rs[p], k);
	}
	maintain(p);
	if( bad(p) ) rebuild(p);
}

int query(int p, int xl, int yl, int xr, int yr) {
	if(!p || xr < L[p] || xl > R[p] || yr < D[p] || yl > U[p]) return 0;
	if(xl <= L[p] && R[p] <= xr && yl <= D[p] && U[p] <= yr) return sum[p];
	int res = 0;
	if(xl <= a[p].x && a[p].x <= xr && yl <= a[p].y && a[p].y <= yr) res += a[p].v;
	return query(ls[p], xl, yl, xr, yr) + query(rs[p], xl, yl, xr, yr) + res;
}

signed main()
{
	int lastans = 0, op, x1, y1, x2, y2, A; 
	scanf("%d", &n); cnt = 0;
	while( scanf("%d", &op) ) {
		if(op == 1) {
			cin >> x1 >> y1 >> A;
			x1 ^= lastans;
			y1 ^= lastans;
			A ^= lastans;
			a[++cnt] = (node){x1, y1, A};
			insert(rt, cnt);
		}
		if(op == 2) {
			cin >> x1 >> y1 >> x2 >> y2;
			x1 ^= lastans;
			y1 ^= lastans;
			x2 ^= lastans;
			y2 ^= lastans;
			printf("%d\n", lastans = query(rt, x1, y1, x2, y2) );
		}
		if(op == 3) break;
	}
	return 0;
}
\end{lstlisting}


\subsection{珂朵莉树}
\begin{lstlisting}
#pragma GCC optimize(2)
#include<bits/stdc++.h>
#define ll long long
using namespace std;

const int MAXN = 1e5 + 5;

namespace ODT {
	struct node {
		ll l, r;
		mutable ll v;
		node(ll l, ll r, ll v) : l(l), r(r), v(v) {}
		bool operator < (const node& a) const { return l < a.l; }
	};
	
	set<node> tree;
	int n, q, sum;
	
	set<node>::iterator split(ll pos) {
		auto it = tree.lower_bound( node(pos, 0, 0) );
		if(it != tree.end() && it->l == pos) return it;
		it--;
		ll l = it->l, r = it->r, v = it->v;
		tree.erase(it);
		tree.insert( node(l, pos - 1, v) );
		return tree.insert( node(pos, r, v) ).first;
	}
	
	void assign(ll l, ll r, ll v) {
		int tot = 0, len = 0;
		auto end = split(r + 1), begin = split(l);
		
		for(auto it = begin; it != end; it++) {
			len += (it->r - it->l + 1);
			tot += it->v * (it->r - it->l + 1);
		}
		
		tree.erase(begin, end);
		tree.insert( node(l, r, v) );
		
		if(v == 1) sum += (len - tot);
		else sum -= tot;
	}
	
	// codeforces 915e
	void solve() {
		cin >> n >> q;
		tree.insert( node(1, n, 1) );
		sum = n;
		while(q--) {
			int l, r, k; cin >> l >> r >> k;
			assign(l, r, k == 1 ? 0 : 1);
			cout << sum << "\n";
		}
	}
}
using namespace ODT;

signed main()
{
	ios::sync_with_stdio(false); cin.tie(0); cout.tie(0);
	ODT::solve();
    return 0;
}
\end{lstlisting}


\subsection{树状数组}
\begin{lstlisting}
int lowbit(int x) { return x & (-x); }

void Modify(int x, int k) {
	while(x <= n) c[x] += k, x += lowbit(x);
} 

int Query(int x) {
	int res = 0;
	while(x > 0) res += c[x], x -= lowbit(x);
	return res;
}

// 树状数组二分 O(logn)求全局第k小
int Kth(int k) {
	int cnt = 0, res = 0;
	for(int i = log2(n); i >= 0; --i) {
		res += (1 << i);
		if(res >= n || cnt + c[res] >= k) res -= (1 << i);
		else cnt += c[res];
	}
	return res + 1;
}
\end{lstlisting}


\subsection{ST表}
\begin{lstlisting}
void Init() {
	for(int k = 1; (1 << k) <= n; k++) {
		for(int i = 1; i <= n; i++) {
			st[i][k] = max(st[i][k - 1], st[i + (1 << (k - 1) )][k - 1]);
		}
	}
}

int Query(int l, int r) {
	int t = floor(log2(r - l + 1));
	return max(st[l][t], st[r - (1 << t) + 1][t]);
}
\end{lstlisting}

\subsection{单调队列}
\begin{lstlisting}
void QueryMin() {
	memset(Index, 0, sizeof(Index)); memset(Value, 0, sizeof(Value));
	int head = 1, tail = 0;
	for(int i = 1; i <= n; i++) {
		while(head <= tail && Index[head] < i - k + 1) head++;
		while(head <= tail && Value[tail] > a[i]) tail--; 
		Index[++tail] = i;
		Value[tail] = a[i];
	}
}

void QueryMax() {
	memset(Index, 0, sizeof(Index)); memset(Value, 0, sizeof(Value));
	int head = 1, tail = 0;
	for(int i = 1; i <= n; i++) {
		while(head <= tail && Index[head] < i - k + 1) head++;
		while(head <= tail && Value[tail] < a[i]) tail--;
		Index[++tail] = i;
		Value[tail] = a[i];
	}
}

void Init() {
	scanf("%d%d", &n, &k);
	for(int i = 1; i <= n; i++) scanf("%d", &a[i]);
	QueryMin(); QueryMax();
}
\end{lstlisting}

\subsection{单调栈}
\begin{lstlisting}
/* 
定义函数f[i]代表数列中第i个元素之后第一个大于a[i]的元素的下标 求(f数组)
*/
void Query() {
	int top = 0;
	for(int i = 1; i <= n; i++) {
		while(top > 0 && Value[top] < a[i]) { //单调栈保持降序
			f[Index[top]] = i;
			top--;
		}
		Value[++top] = a[i];
		Index[top] = i;
	}
}
\end{lstlisting}


\subsection{块状数据结构}

\subsubsection{分块}
\begin{lstlisting}
#include<bits/stdc++.h>
using namespace std;
const int MAXN = (int)50005;
int a[MAXN], tag[MAXN], ValueAllOne[MAXN], id[MAXN], sum[MAXN], n, S;
vector<int> b[MAXN];
pair<int, int> block[MAXN];
 
void Init() {
	S = sqrt(n);
	memset(id, -1, sizeof(id));
	for(int i = 1; i <= n; i++) id[i] = (i - 1) / S + 1;
	
	int j = 1;
	for(int i = id[1]; i <= id[n]; i++) {
		// 维护区间和 
		block[i].first = j;
		for(; id[j] == i; j++) sum[i] += a[j];
		block[i].second = j - 1;
		
		// 维护区间排序 
		for(; id[j] == i; j++) b[i].push_back(a[j]);
		sort(b[i].begin(), b[i].end());		
	}
}

void PushDownSqrt(int cid) {
	int l = block[cid].first, r = block[cid].second;
	if(ValueAllOne[cid] == 1 || tag[cid] == 0) return ;
	
	bool flag = true;
	for(int i = l; i <= r; i++) {
		for(int j = 1; j <= tag[cid]; j++) {
			if(a[i] == 1) break;
			a[i] = floor(sqrt(1.0 * a[i]));
		}
		if(a[i] > 1) flag = false;
	}
	
	tag[cid] = 0;
	if(flag) ValueAllOne[cid] = 1;
	sum[cid] = 0;
	for(int i = l; i <= r; i++) sum[cid] += a[i];
}

void ModifySqrt(int l, int r) {	// 区间元素都开根号 
	int lid = id[l], rid = id[r];
	if(lid == rid) {
		SqrtPushDown(lid);
		for(int i = l; i <= r; i++) {
			int cur = floor(sqrt(1.0 * a[i]));
			sum[lid] -= a[i], sum[lid] += cur;
			a[i] = cur;
		}
	} else {
		PushDown(lid); PushDown(rid);
		for(int i = l; id[i] == lid; i++) {
			int cur = floor(sqrt(1.0 * a[i]));
			sum[lid] -= a[i], sum[lid] += cur;
			a[i] = cur;			
		}
		for(int i = r; id[i] == rid; i--) {
			int cur = floor(sqrt(1.0 * a[i]));
			sum[rid] -= a[i], sum[rid] += cur;
			a[i] = cur;	
		}
		for(int i = lid + 1; i < rid; i++) ++tag[i];
	}
}

int QuerySum(int l, int r) {	// 区间求和 
	int lid = id[l], rid = id[r];
	if(lid == rid) {
		PushDown(lid);
		int res = 0;
		for(int i = l; i <= r; i++) res += a[i];
		return res;
	} else {
		int res = 0;
		PushDown(lid); PushDown(rid);
		for(int i = l; id[i] == lid; i++) res += a[i];
		for(int i = r; id[i] == rid; i--) res += a[i];
		for(int i = lid + 1; i < rid; i++) PushDown(i), res += sum[i];
		return res;
	}
}

void ModifyAdd(int l, int r, int c) { // 区间元素都加上c 
	int lid = id[l], rid = id[r];
	if(lid == rid) {
		b[lid].clear();
		for(int i = l; i <= r; i++) {
			a[i] += c;
			b[lid].push_back(a[i]);	
		}
		for(int i = l - 1; id[i] == lid; i--) b[lid].push_back(a[i]);
		for(int i = r + 1; id[i] == lid; i++) b[lid].push_back(a[i]);
		sort(b[lid].begin(), b[lid].end());
	} else {
		b[lid].clear();
		for(int i = l; id[i] == lid; i++) {
			a[i] += c;
			b[lid].push_back(a[i]);
		}
		for(int i = l - 1; id[i] == lid; i--) b[lid].push_back(a[i]);
		sort(b[lid].begin(), b[lid].end());
		
		b[rid].clear();
		for(int i = r; id[i] == rid; i--) {
			a[i] += c;
			b[rid].push_back(a[i]);
		}
		for(int i = r + 1; id[i] == rid; i++) b[rid].push_back(a[i]);
		sort(b[rid].begin(), b[rid].end());
		
		for(int i = lid + 1; i < rid; i++) tag[i] += c;
	}
}

int QuerySumOfSmaller(int l, int r, int c) { // 区间查询小于c的数字个数 
	int lid = id[l], rid = id[r];
	if(lid == rid) {
		int res = 0;
		for(int i = l; i <= r; i++) {
			res += (a[i] + tag[lid] < c);
		}
		return res;
	} else {
		int res = 0;
		for(int i = l; id[i] == lid; i++) {
			res += (a[i] + tag[lid] < c);
		}
		for(int i = r; id[i] == rid; i--) {
			res += (a[i] + tag[rid] < c);
		}
		for(int i = lid + 1; i < rid; i++) {
			res += lower_bound(b[i].begin(), b[i].end(), c - tag[i]) - b[i].begin();
		}
		return res;
	}
}

int QueryPre(int l, int r, int c) { // 区间内查询c的前驱(比其小的最大元素)
	int lid = id[l], rid = id[r];
	if(lid == rid) {
		int res = -INF;
		for(int i = l; i <= r; i++) {
			if(a[i] + tag[lid] < c) res = max(res, a[i] + tag[lid]);
		}
		return res;
	} else {
		int res = -INF;
		for(int i = l; id[i] == lid; i++) {
			if(a[i] + tag[lid] < c) res = max(res, a[i] + tag[lid]);
		}
		for(int i = r; id[i] == rid; i--) {
			if(a[i] + tag[rid] < c) res = max(res, a[i] + tag[rid]);
		}
		for(int i = lid + 1; i < rid; i++) {
			int cur = lower_bound(b[i].begin(), b[i].end(), c - tag[i]) - b[i].begin() - 1;
			if(cur >= 0) res = max(res, b[i][cur] + tag[i]);
		}
		return res;	
	}
}

signed main()
{	
	Init();
	//Solve();
	return 0;
}
\end{lstlisting}

\subsubsection{分块求区间众数}
\begin{lstlisting}
//#pragma GCC optimize(2)
#include<bits/stdc++.h>
#define int long long
#define pir make_pair
#define pii pair<int, int>
#define fi first
#define se second
using namespace std;
const int MAXN = 40005;
const int MAXS = 205;
const int MOD = 1e9 + 7;

int n, m, s, a[MAXN], val[MAXN], id[MAXN], l[MAXS], r[MAXS], c[MAXN], vis[MAXN], sum[MAXS][MAXN], ans[MAXS][MAXS];
map<int, int> Map;

void Init() {
	int cnt = 0;
	for(auto &i : Map) i.se = ++cnt, val[i.se] = i.fi;
	for(int i = 1; i <= n; i++) a[i] = Map[ a[i] ];

	s = sqrt(n);
	for(int i = 1; i <= n; i++) id[i] = (i - 1) / s + 1, r[ id[i] ] = i;
	for(int i = n; i >= 1; i--) l[ id[i] ] = i;
	
	for(int i = 1; i <= n; i++) sum[ id[i] ][ a[i] ]++;
	for(int i = 1; i <= id[n]; i++) for(int j = 1; j <= n; j++) sum[i][j] += sum[i - 1][j];	
	
	for(int i = 1; i <= id[n]; i++) {
		int res = 0;
		for(int j = i; j >= 1; j--) {
			for(int k = l[j]; k <= r[j]; k++) {
				c[ a[k] ]++;
				if(c[ a[k] ] > c[res] || (c[ a[k] ] == c[res] && a[k] < res) ) res = a[k];
			}
			ans[j][i] = res;
		}
		for(int j = 1; j <= n; j++) c[j] = 0;
	}
}

int Query(int L, int R) {
	int lid = id[L], rid = id[R], res = 0;
	if(rid - lid + 1 <= 2) {
		res = 0;
		for(int i = L; i <= R; i++) {
			c[ a[i] ]++;
			if(c[ a[i] ] > c[res] || (c[ a[i] ] == c[res] && a[i] < res) ) res = a[i];
		}
		for(int i = L; i <= R; i++) c[ a[i] ] = 0;
	} else {
		res = ans[lid + 1][rid - 1];
		c[res] += sum[rid - 1][res] - sum[lid][res];
		vis[res] = 1;
		
		for(int i = L; i <= r[lid]; i++) c[ a[i] ]++;
		for(int i = l[rid]; i <= R; i++) c[ a[i] ]++;
		
		for(int i = L; i <= r[lid]; i++) {
			if(!vis[ a[i] ]) {
				c[ a[i] ] += sum[rid - 1][ a[i] ] - sum[lid][ a[i] ];
				vis[ a[i] ] = 1;
				if(c[ a[i] ] > c[res] || (c[ a[i] ] == c[res] && a[i] < res) ) res = a[i];
			}
		}
		for(int i = l[rid]; i <= R; i++) {
			if(!vis[ a[i] ]) {
				c[ a[i] ] += sum[rid - 1][ a[i] ] - sum[lid][ a[i] ];
				vis[ a[i] ] = 1;
				if(c[ a[i] ] > c[res] || (c[ a[i] ] == c[res] && a[i] < res) ) res = a[i];
			}
		}
		
		for(int i = L; i <= r[lid]; i++) c[ a[i] ] = vis[ a[i] ] = 0;
		for(int i = l[rid]; i <= R; i++) c[ a[i] ] = vis[ a[i] ] = 0;
		c[ ans[lid + 1][rid - 1] ] = vis[ ans[lid + 1][rid - 1] ] = 0;
	}
	return res;
}

signed main()
{
	ios::sync_with_stdio(false); cin.tie(0); cout.tie(0);
	cin >> n >> m;
	for(int i = 1; i <= n; i++) cin >> a[i], Map[ a[i] ] = 1;
	Init();
	
	int x = 0;
	while(m--) {
		int l, r; cin >> l >> r;
		l = (l + x - 1) % n + 1, r = (r + x - 1) % n + 1;
		if(l > r) swap(l, r);
		x = val[ Query(l, r) ];
		cout << x << "\n";
	}
	return 0;
}
\end{lstlisting}

\subsubsection{块状链表}
\begin{lstlisting}
//#pragma GCC optimize(2)
#include<bits/stdc++.h>
using namespace std;
const int MAXN = 2010, MAXM = 2010;

struct Node {
	char ch[MAXM + 5];
	int sze, lnode, rnode;	
	void push_back(char c) { ch[sze++] = c; }
}p[MAXN + 5];

int pos1 = 0, pos2 = 0;	//光标在块中/块内位置
int id[MAXN + 5], idx = 0; //可分配的编号池

void Add(int u, int v) {	//将节点v插到节点u的右边
	p[v].rnode = p[u].rnode; 
	p[ p[v].rnode ].lnode = v;
	p[u].rnode = v; 
	p[v].lnode = u;
}

void Del(int u) {	//删除节点u
	p[ p[u].lnode ].rnode = p[u].rnode;
	p[ p[u].rnode ].lnode = p[u].lnode;
	p[u].lnode = p[u].rnode = p[u].sze = 0;
	id[++idx] = u;
}

void Merge() {
	for(int k = p[0].rnode; k; k = p[k].rnode) {
		while(p[k].rnode && p[k].sze + p[ p[k].rnode ].sze < MAXM) {
			int rnode = p[k].rnode;
			if(pos1 == rnode) pos1 = k, pos2 += p[k].sze;	// 与下一条语句顺序不能调换
			for(int i = 0; i < p[rnode].sze; i++) p[k].push_back( p[rnode].ch[i] ); 
			Del(rnode);
		}
	}
}

void Move(int k) {	//移动到第k个字符后面
	pos1 = p[0].rnode;
	while (k > p[pos1].sze) k -= p[pos1].sze, pos1 = p[pos1].rnode;
	pos2 = k - 1;
}

void Insert(string S, int n) {	//在光标后面插入字符串S, 长度为n
	if(pos2 + 1 != p[pos1].sze) {	//分裂
		int u = id[idx--];
		for(int i = pos2 + 1; i < p[pos1].sze; i++) p[u].push_back( p[pos1].ch[i] );
		p[pos1].sze = pos2 + 1;
		Add(pos1, u);
	}
	int cur = pos1, i = 0;
	while(i < n) {
		int u = id[idx--];
		for(; i < n && p[u].sze < MAXN; i++) p[u].push_back(S[i]);
		Add(cur, u);
		cur = u;
	}
	Merge();
}

void Delete(int n) {	//删除光标后的n个字符
	if(pos2 + 1 + n <= p[pos1].sze) {
		for(int i = pos2 + 1, j = pos2 + 1 + n; j < p[pos1].sze; i++, j++) {
			p[pos1].ch[i] = p[pos1].ch[j];
		}
		p[pos1].sze -= n;
	} else {
		n -= (p[pos1].sze - pos2 - 1);
		p[pos1].sze = pos2 + 1;
		while(p[pos1].rnode && n >= p[ p[pos1].rnode ].sze) {
			n -= p[ p[pos1].rnode ].sze;
			Del(p[pos1].rnode);
		}
		int u = p[pos1].rnode;
		for(int i = 0, j = n; j < p[u].sze; i++, j++) {
			p[u].ch[i] = p[u].ch[j];
		}
		p[u].sze -= n;
	}
	Merge();
}

void Get(int n) {	//获取光标后n个字母
	if(pos2 + 1 + n <= p[pos1].sze) {
		for(int i = pos2 + 1; i <= pos2 + n; i++) cout << p[pos1].ch[i];
	} else {
		n -= (p[pos1].sze - pos2 - 1);
		for(int i = pos2 + 1; i < p[pos1].sze; i++) cout << p[pos1].ch[i];
		int cur = pos1;
		while(p[cur].rnode && n >= p[ p[cur].rnode ].sze) {
			n -= p[ p[cur].rnode ].sze;
			for(int i = 0; i < p[ p[cur].rnode ].sze; i++) cout << p[ p[cur].rnode ].ch[i];
			cur = p[cur].rnode;
		}
		int u = p[cur].rnode;
		for(int i = 0; i < n; i++) cout << p[u].ch[i];
	}
	cout << "\n";
}

void Prev() {	//光标前移
	if(pos2) pos2--;
	else pos1 = p[pos1].lnode, pos2 = p[pos1].sze - 1; 
}

void Next() {	//光标后移
	if(pos2 != p[pos1].sze - 1) pos2++;
	else pos1 = p[pos1].rnode, pos2 = 0;
}

void Solve() {
	for(int i = 1; i <= MAXN; i++) id[++idx] = i;
	Insert("\n", 1); Move(1);	//预防越界
	int t; cin >> t;
	while(t--) {
		string op, S;
		int n;
		cin >> op;
		if(op == "Insert") {
			cin >> n;
			int cur = 0; S = "";
			while(cur < n) {
				char ch = getchar();
				if(32 <= ch && ch <= 126) S.push_back(ch), cur++;
			}
			Insert(S, n);
		} 
		else if(op == "Move") cin >> n, Move(n + 1);
		else if(op == "Delete") cin >> n, Delete(n);
		else if(op == "Get") cin >> n, Get(n);
		else if(op == "Prev") Prev();
		else if(op == "Next") Next();	
	}
}

signed main() { Solve(); return 0; }
\end{lstlisting}


\subsubsection{块状数组}
\begin{lstlisting}
#include<bits/stdc++.h>
using namespace std;
const int MAXN = (int)50005;
int a[MAXN], tag[MAXN], ValueAllOne[MAXN], id[MAXN], sum[MAXN], n, S;
vector<int> b[MAXN];
pair<int, int> block[MAXN];
 
void Init() {
	S = sqrt(n);
	memset(id, -1, sizeof(id));
	for(int i = 1; i <= n; i++) id[i] = (i - 1) / S + 1;
	
	int j = 1;
	for(int i = id[1]; i <= id[n]; i++) {
		// 维护区间和
		block[i].first = j;
		for(; id[j] == i; j++) sum[i] += a[j];
		block[i].second = j - 1;
		
		// 维护区间排序
		for(; id[j] == i; j++) b[i].push_back(a[j]);
		sort(b[i].begin(), b[i].end());		
	}
}

void PushDownSqrt(int cid) {
	int l = block[cid].first, r = block[cid].second;
	if(ValueAllOne[cid] == 1 || tag[cid] == 0) return ;
	
	bool flag = true;
	for(int i = l; i <= r; i++) {
		for(int j = 1; j <= tag[cid]; j++) {
			if(a[i] == 1) break;
			a[i] = floor(sqrt(1.0 * a[i]));
		}
		if(a[i] > 1) flag = false;
	}
	
	tag[cid] = 0;
	if(flag) ValueAllOne[cid] = 1;
	sum[cid] = 0;
	for(int i = l; i <= r; i++) sum[cid] += a[i];
}

void ModifySqrt(int l, int r) {	// 区间元素开根号
	int lid = id[l], rid = id[r];
	if(lid == rid) {
		SqrtPushDown(lid);
		for(int i = l; i <= r; i++) {
			int cur = floor(sqrt(1.0 * a[i]));
			sum[lid] -= a[i], sum[lid] += cur;
			a[i] = cur;
		}
	} else {
		PushDown(lid); PushDown(rid);
		for(int i = l; id[i] == lid; i++) {
			int cur = floor(sqrt(1.0 * a[i]));
			sum[lid] -= a[i], sum[lid] += cur;
			a[i] = cur;			
		}
		for(int i = r; id[i] == rid; i--) {
			int cur = floor(sqrt(1.0 * a[i]));
			sum[rid] -= a[i], sum[rid] += cur;
			a[i] = cur;	
		}
		for(int i = lid + 1; i < rid; i++) ++tag[i];
	}
}

int QuerySum(int l, int r) {	// 区间求和
	int lid = id[l], rid = id[r];
	if(lid == rid) {
		PushDown(lid);
		int res = 0;
		for(int i = l; i <= r; i++) res += a[i];
		return res;
	} else {
		int res = 0;
		PushDown(lid); PushDown(rid);
		for(int i = l; id[i] == lid; i++) res += a[i];
		for(int i = r; id[i] == rid; i--) res += a[i];
		for(int i = lid + 1; i < rid; i++) PushDown(i), res += sum[i];
		return res;
	}
}

void ModifyAdd(int l, int r, int c) { // 区间元素都+c
	int lid = id[l], rid = id[r];
	if(lid == rid) {
		b[lid].clear();
		for(int i = l; i <= r; i++) {
			a[i] += c;
			b[lid].push_back(a[i]);	
		}
		for(int i = l - 1; id[i] == lid; i--) b[lid].push_back(a[i]);
		for(int i = r + 1; id[i] == lid; i++) b[lid].push_back(a[i]);
		sort(b[lid].begin(), b[lid].end());
	} else {
		b[lid].clear();
		for(int i = l; id[i] == lid; i++) {
			a[i] += c;
			b[lid].push_back(a[i]);
		}
		for(int i = l - 1; id[i] == lid; i--) b[lid].push_back(a[i]);
		sort(b[lid].begin(), b[lid].end());
		
		b[rid].clear();
		for(int i = r; id[i] == rid; i--) {
			a[i] += c;
			b[rid].push_back(a[i]);
		}
		for(int i = r + 1; id[i] == rid; i++) b[rid].push_back(a[i]);
		sort(b[rid].begin(), b[rid].end());
		
		for(int i = lid + 1; i < rid; i++) tag[i] += c;
	}
}

int QuerySumOfSmaller(int l, int r, int c) { // 区间查询小于c的数字个数
	int lid = id[l], rid = id[r];
	if(lid == rid) {
		int res = 0;
		for(int i = l; i <= r; i++) {
			res += (a[i] + tag[lid] < c);
		}
		return res;
	} else {
		int res = 0;
		for(int i = l; id[i] == lid; i++) {
			res += (a[i] + tag[lid] < c);
		}
		for(int i = r; id[i] == rid; i--) {
			res += (a[i] + tag[rid] < c);
		}
		for(int i = lid + 1; i < rid; i++) {
			res += lower_bound(b[i].begin(), b[i].end(), c - tag[i]) - b[i].begin();
		}
		return res;
	}
}

int QueryPre(int l, int r, int c) { // 区间内查询c的前驱(比其小的最大元素)
	int lid = id[l], rid = id[r];
	if(lid == rid) {
		int res = -INF;
		for(int i = l; i <= r; i++) {
			if(a[i] + tag[lid] < c) res = max(res, a[i] + tag[lid]);
		}
		return res;
	} else {
		int res = -INF;
		for(int i = l; id[i] == lid; i++) {
			if(a[i] + tag[lid] < c) res = max(res, a[i] + tag[lid]);
		}
		for(int i = r; id[i] == rid; i--) {
			if(a[i] + tag[rid] < c) res = max(res, a[i] + tag[rid]);
		}
		for(int i = lid + 1; i < rid; i++) {
			int cur = lower_bound(b[i].begin(), b[i].end(), c - tag[i]) - b[i].begin() - 1;
			if(cur >= 0) res = max(res, b[i][cur] + tag[i]);
		}
		return res;	
	}
}

signed main()
{	
	Init();
	//Solve();
	return 0;
}
\end{lstlisting}

\subsubsection{莫队}
\begin{lstlisting}
#include<bits/stdc++.h>
//#pragma GCC optimize(2)

using namespace std;

const int MAXN = 50000 + 5;
int BLOCK_SIZE;

struct Query {
	int l, r, id;
	bool operator < (const Query &a) const {
		if(l / BLOCK_SIZE != a.l / BLOCK_SIZE) return l < a.l;
		return (l / BLOCK_SIZE) & 1 ? r < a.r : r > a.r;
	}
}q[MAXN];

int n, m, col[MAXN], cnt[MAXN];
long long sum = 0;
pair<long long, long long> ans[MAXN];

void Update(int c, int num) {
	sum -= 1ll * cnt[c] * (cnt[c] - 1) / 2;
	cnt[c] += num;
	sum += 1ll * cnt[c] * (cnt[c] - 1) / 2;
}

signed main()
{
	ios::sync_with_stdio(false); cin.tie(0);
	cin >> n >> m;
	BLOCK_SIZE = (int)ceil( sqrt(1.0 * n) );
	for(int i = 1; i <= n; i++) cin >> col[i];
	for(int i = 1; i <= m; i++) cin >> q[i].l >> q[i].r, q[i].id = i;
	sort(q + 1, q + m + 1);
	for(int i = 1, l = 1, r = 0; i <= m; i++) {
		if(q[i].l == q[i].r) {
			ans[ q[i].id ] = make_pair(0ll, 1ll);
			continue;
		}
		while(l > q[i].l) Update(col[--l], 1);
		while(r < q[i].r) Update(col[++r], 1);
		while(l < q[i].l) Update(col[l++], -1);
		while(r > q[i].r) Update(col[r--], -1);
		long long tot = 1ll * (r - l + 1) * (r - l) / 2;
		ans[ q[i].id ] = make_pair(sum, tot);
	}
	for(int i = 1; i <= m; i++) {
		if(ans[i].first) {
			long long g = __gcd(ans[i].first, ans[i].second);
			ans[i].first /= g, ans[i].second /= g;
		} else ans[i].second = 1;
		cout << ans[i].first << "/" << ans[i].second << "\n";
	}
	return 0;
}
\end{lstlisting}


\subsection{线段树}
\begin{lstlisting}
/* 区间加法 区间乘法 区间求和 */

#include<bits/stdc++.h>
#define ll long long
using namespace std;
const int MAXN = (int)1e5 + 5;

int n, m, MOD;
ll mul[MAXN << 2], add[MAXN << 2], sum[MAXN << 2], a[MAXN];

inline int ls(int x) { return x << 1; }
inline int rs(int x) { return x << 1 | 1; }

void PushUp(int p) {
	sum[p] = (sum[ls(p)] + sum[rs(p)]) % MOD;
}

void PushDown(int p, int l, int r) {
	int mid = l + r >> 1;
	mul[ls(p)] = mul[ls(p)] * mul[p] % MOD;
	mul[rs(p)] = mul[rs(p)] * mul[p] % MOD;
	add[ls(p)] = (add[ls(p)] * mul[p] % MOD + add[p]) % MOD;
	add[rs(p)] = (add[rs(p)] * mul[p] % MOD + add[p]) % MOD;
	sum[ls(p)] = (sum[ls(p)] * mul[p] % MOD + add[p] * (mid - l + 1) % MOD) % MOD;
	sum[rs(p)] = (sum[rs(p)] * mul[p] % MOD + add[p] * (r - mid) % MOD) % MOD;	
	add[p] = 0, mul[p] = 1;
}

void Build(int l, int r, int p) {
	add[p] = 0, mul[p] = 1; // significant
	if(l == r) {
		sum[p] = a[l];
		return ;
	}
	int mid = l + r >> 1;
	Build(l, mid, ls(p));
	Build(mid + 1, r, rs(p));
	PushUp(p);
}

void ModifyAdd(int nl, int nr, int l, int r, int p, ll k) {
	if(nl <= l && nr >= r) {
		add[p] = (add[p] + k) % MOD;
		sum[p] = (sum[p] + 1ll * (r - l + 1) * k % MOD) % MOD;
		return ;
	}
	PushDown(p, l, r);
	int mid = l + r >> 1;
	if(nl <= mid) ModifyAdd(nl, nr, l, mid, ls(p), k);
	if(nr > mid) ModifyAdd(nl, nr, mid + 1, r, rs(p), k);
	PushUp(p);
}

void ModifyMul(int nl, int nr, int l, int r, int p, ll k) {
	if(nl <= l && nr >= r) {
		mul[p] = mul[p] * k % MOD;
		add[p] = add[p] * k % MOD;
		sum[p] = sum[p] * k % MOD;
		return ;
	}
	PushDown(p, l, r);
	int mid = l + r >> 1;
	if(nl <= mid) ModifyMul(nl, nr, l, mid, ls(p), k);
	if(nr > mid) ModifyMul(nl, nr, mid + 1, r, rs(p), k);
	PushUp(p);
}

ll Query(int nl, int nr, int l, int r, int p) {
	if(nl <= l && nr >= r) return sum[p];
	PushDown(p, l, r);
	int mid = l + r >> 1, res = 0;
	if(nl <= mid) res = (res + Query(nl, nr, l, mid, ls(p)) ) % MOD;
	if(nr > mid) res = (res + Query(nl, nr, mid + 1, r, rs(p)) ) % MOD;
	return res;
}

signed main()
{
	scanf("%d%d%d", &n, &m, &MOD);
	for(int i = 1; i <= n; i++) scanf("%lld", &a[i]);
	Build(1, n, 1);
	while(m--) {
		int op, l, r; ll k;
		scanf("%d%d%d", &op, &l, &r);
		if(op == 1) {
			scanf("%lld", &k);
			ModifyMul(l, r, 1, n, 1, k);
		} else if(op == 2) {
			scanf("%lld", &k);
			ModifyAdd(l, r, 1, n, 1, k);
		} else {
			printf("%lld\n", Query(l, r, 1, n, 1) );
		}
		
	}
	return 0;
}
\end{lstlisting}

\subsubsection{线段树合并分裂}
\begin{lstlisting}
// 对每个点开一棵线段树然后能维护的东西,线段树合并都能维护
#pragma GCC optimize(2)
#include<bits/stdc++.h>
#define int long long
#define pir make_pair
#define pii pair<int, int>
#define fi first
#define se second
using namespace std;
const int MAXN = 1e5 + 5;
const int MOD = 1e9 + 7;

int n, m, Op[MAXN];
set<int> Set;

struct Node {
	int ltype, rtype, ans, sze;
	friend Node operator + (Node a, Node b) {
		if(!a.sze) return b;
		if(!b.sze) return a;
		Node res;
		res.ans = a.ans + b.ans + (a.rtype ^ b.ltype);
		res.ltype = a.ltype;
		res.rtype = b.rtype;
		res.sze = a.sze + b.sze;
		return res;
	}
}tree[MAXN << 2];

struct Segment_Tree {
	int root[MAXN * 40], son[MAXN * 40][3];
	Node sum[MAXN * 40];
	int pool[MAXN * 40], delcnt = 0, cnt = 0;
	
	void PushUp(int p) { sum[p] = sum[ son[p][0] ] + sum[ son[p][1] ]; }
	
	int NewNode() { return delcnt ? pool[delcnt--] : ++cnt; }
	
	void DelNode(int p) {
		pool[++delcnt] = p;
		sum[p] = {0, 0, 0, 0};
		son[p][0] = son[p][1] = 0;
	}
	
	void Insert(int& p, int l, int r, int loc) {
		if(!p) p = NewNode();
		if(l == r) {
			sum[p].ltype = sum[p].rtype = (loc & 1);
			sum[p].ans = 0, sum[p].sze = 1;
			return ;
		}
		int mid = l + r >> 1;
		if(loc <= mid) Insert(son[p][0], l, mid, loc);
		else Insert(son[p][1], mid + 1, r, loc);
		PushUp(p);
	}
	
	int Merge(int u, int v, int l = 1, int r = n) {
		if(!u || !v) return u + v;
		if(l == r) {
			sum[u] = sum[u] + sum[v];
			DelNode(v); 
			return u;
		}
		int mid = l + r >> 1;
		son[u][0] = Merge(son[u][0], son[v][0], l, mid);
		son[u][1] = Merge(son[u][1], son[v][1], mid + 1, r);
		DelNode(v);
		PushUp(u);
		return u;
	}
	
	void Split(int u, int& v, int k, int flag) {	//把u节点分裂,得到新的放到v里面,分裂前k个数的节点
		if(!k) return ;
		v = NewNode();
		if(k >= sum[ son[u][flag] ].sze) {
			Split(son[u][flag ^ 1], son[v][flag ^ 1], k - sum[ son[u][flag] ].sze, flag);
			swap(son[u][flag], son[v][flag]);
		} else Split(son[u][flag], son[v][flag], k, flag);
		PushUp(u), PushUp(v);
	}
}S;

void Modify(int loc, int l, int r, int p) {
	if(l == r) {
		tree[p] = S.sum[ S.root[loc] ];
		if(Op[loc]) swap(tree[p].ltype, tree[p].rtype);
		return ;
	}
	int mid = l + r >> 1;
	if(loc <= mid) Modify(loc, l, mid, p << 1);
	else Modify(loc, mid + 1, r, p << 1 | 1);
	tree[p] = tree[p << 1] + tree[p << 1 | 1];
}

Node Query(int nl, int nr, int l, int r, int p) {
	if(nl <= l && nr >= r) return tree[p];
	int mid = l + r >> 1;
	if(nl > mid) return Query(nl, nr, mid + 1, r, p << 1 | 1);
	else if(nr <= mid) return Query(nl, nr, l, mid, p << 1);
	else return Query(nl, nr, l, mid, p << 1) + Query(nl, nr, mid + 1, r, p << 1 | 1);
}

set<int>::iterator Split(int x) {
	auto pos = Set.lower_bound(x), tmp = pos;
	if(*pos == x) return pos;
	pos--;
	S.Split(S.root[*pos], S.root[x], *tmp - x, Op[*pos] ^ 1);
	Modify(*pos, 1, n, 1);
	Op[x] = Op[*pos];
	Modify(x, 1, n, 1);
	Set.insert(x);
	return Set.lower_bound(x);
}

signed main()
{
	ios::sync_with_stdio(false); cin.tie(0); cout.tie(0);
	cin >> n >> m; Set.insert(n + 1);
	for(int i = 1; i <= n; i++) {
		int x; cin >> x;
		Set.insert(i);
		S.Insert(S.root[i], 1, n, x);
		Modify(i, 1, n, 1);
	}
	while(m--) {
		int op, l, r; cin >> op >> l >> r;
		auto pl = Split(l), pr = Split(r + 1);
		if(op == 3) {
			cout << Query(l, r, 1, n, 1).ans + 1 << "\n";
		} else {
			
			for(auto i = ++pl; i != pr; i++) {
				S.Merge(S.root[l], S.root[*i]);
				S.root[*i] = 0;
				Modify(*i, 1, n, 1);
			}
			Set.erase(pl, pr);
			Op[l] = op - 1;
			Modify(l, 1, n, 1);
		}
	}
	return 0;
}
\end{lstlisting}

\subsubsection{扫描线}
\begin{lstlisting}
// 求n个矩形的面积并
#include<bits/stdc++.h>
#define int long long
using namespace std;
const int MAXN = (int)1e5 + 5;

struct ScanLine {
	int x, lowy, highy, io;	//io记录 入边/出边 
	ScanLine(){}
	ScanLine(int x, int y1, int y2, int io) : x(x), lowy(y1), highy(y2), io(io){}
	bool operator < (const ScanLine &a) const { return x < a.x; }
}line[MAXN << 1];

int n, ans = 0, tot, cnt, yy[MAXN << 1];
int length[MAXN << 3], tag[MAXN << 3];

int ls(int x) { return x << 1; }
int rs(int x) { return x << 1 | 1; }

void PushUp(int p, int l, int r) {
	if(tag[p]) length[p] = yy[r] - yy[l];
	else if(l + 1 == r) length[p] = 0;
	else length[p] = length[ls(p)] + length[rs(p)];
}

void Build(int l, int r, int p) {	//注意叶子节点的结构(用区间表示 如[1,2] [2,3]) 
	if(l >= r) return ;
	tag[p] = length[p] = 0;
	if(l + 1 == r) return ;
	int mid = (l + r) >> 1;
	Build(l, mid, ls(p));
	Build(mid, r, rs(p));
}

void Update(int nl, int nr, int l, int r, int p, int io) {
	if(nl > r || nr < l) return ;
	if(nl <= l && nr >= r) {
		tag[p] += io;
		PushUp(p, l, r);
		return ;
	}
	if(l + 1 == r) return ;
	int mid = (l + r) >> 1;
	if(nl <= mid) Update(nl, nr, l, mid, ls(p), io);
	if(nr >= mid) Update(nl, nr, mid, r, rs(p), io);
	PushUp(p, l, r);
}

signed main()
{
	scanf("%lld", &n);
	cnt = 0;
	for(int i = 1; i <= n; i++) {
		int x1, y1, x2, y2;
		scanf("%lld%lld%lld%lld", &x1, &y1, &x2, &y2);	//(x1, y1)为左下角坐标, (x2, y2)为右上角坐标 
		line[++cnt] = ScanLine(x1, y1, y2, 1);
		yy[cnt] = y1;
		line[++cnt] = ScanLine(x2, y1, y2, -1);
		yy[cnt] = y2;
	}
	sort(yy + 1, yy + 1 + cnt);
	sort(line + 1, line + 1 + cnt);
	tot = unique(yy + 1, yy + 1 + cnt) - (yy + 1);	// 离散化, tot记录去重后共多少y值 
	ans = 0;
	Build(1, tot, 1);
	for(int i = 1; i <= cnt; i++) {
		ans += length[1] * (line[i].x - line[i - 1].x);
		int yl = lower_bound(yy + 1, yy + 1 + tot, line[i].lowy) - yy;
		int yr = lower_bound(yy + 1, yy + 1 + tot, line[i].highy) - yy;
		Update(yl, yr, 1, tot, 1, line[i].io);
	} 
	printf("%lld", ans);
	return 0;
}
\end{lstlisting}

\subsubsection{主席树}
\begin{lstlisting}
/*
1. 在某个历史版本上修改某一个位置上的值
2. 访问某个历史版本上的某一位置的值 
*/

#include<bits/stdc++.h>
using namespace std;
const int MAXN = (int)1e6 + 5;
struct Node {
	int ls, rs, val;
}tree[MAXN << 5];
int n, m, a[MAXN], cnt, root[MAXN];

int Build(int l, int r) {
	int dir = ++cnt;
	if(l == r) {
		tree[dir].val = a[l];
		return dir;
	}
	int mid = (l + r) >> 1; 
	tree[dir].ls = Build(l, mid);
	tree[dir].rs = Build(mid + 1, r);
	return dir;
}

int Modify(int p, int l, int r, int loc, int val) {
	int dir = ++cnt;
	tree[dir] = tree[p];
	if(l == r) {
		tree[dir].val = val;
		return dir;
	}
	int mid = (l + r) >> 1;
	if(loc <= mid) tree[dir].ls = Modify(tree[dir].ls, l, mid, loc, val);
	if(loc > mid) tree[dir].rs = Modify(tree[dir].rs, mid + 1, r, loc, val); 
	return dir;
}

int Query(int p, int l, int r, int loc) {
	if(l == r) return tree[p].val;
	int mid = (l + r) >> 1;
	if(loc <= mid) return Query(tree[p].ls, l, mid, loc);
	if(loc > mid) return Query(tree[p].rs, mid + 1, r, loc);
}

signed main()
{
	scanf("%lld%lld", &n, &m);
	for(int i = 1; i <= n; i++) scanf("%lld", &a[i]);
	cnt = 0;
	root[0] = Build(1, n);
	for(int i = 1; i <= m; i++) {
		int v, op, loc, val;
		scanf("%lld%lld%lld", &v, &op, &loc);
		if(op == 1) {
			scanf("%lld", &val);
			root[i] = Modify(root[v], 1, n, loc, val);
		} else {
			printf("%lld\n", Query(root[v], 1, n, loc));
			root[i] = root[v];
		}
	}
	return 0;
}
\end{lstlisting}
\subsubsection{主席树求静态区间第k小}
\begin{lstlisting}
/*
对权值线段树可持久化 再通过树上二分的方式找到[l,r]内第k小值 
*/
#include<bits/stdc++.h>
using namespace std;
const int MAXN = (int)2e5 + 5;

struct Query {
	int x, y, l, r, k;
}q[MAXN];

void Init() {	// 离散化 
	sort(b + 1, b + 1 + tot);
	int cnt = 0;
	a[0].val = -0x3f3f3f3f;	// significant 
	for(int i = 1; i <= n; i++) {
		if(a[i].val != a[i - 1].val) ++cnt;
		a[i].hval = cnt;
		Hash[cnt] = a[i].val;
	}
	sort(a + 1, a + 1 + n, cmp2);
}

int Build(int l, int r) {
	int dir = ++cnt;
	tree[dir].val = 0;
	if(l == r) return dir;
	int mid = l + r >> 1;
	tree[dir].ls = Build(l, mid);
	tree[dir].rs = Build(mid + 1, r);
	return dir;
}

int Modify(int p, int l, int r, int loc) {
	int dir = ++cnt;
	tree[dir] = tree[p];
	tree[dir].val++;
	if(l == r) return dir;
	int mid = l + r >> 1;
	if(loc <= mid) tree[dir].ls = Modify(tree[dir].ls, l, mid, loc); 
	if(loc > mid) tree[dir].rs = Modify(tree[dir].rs, mid + 1, r, loc);
	return dir;
}

int Query(int dl, int dr, int l, int r, int k) { 
	if(l == r) return l;
	int mid = l + r >> 1;
	int x = tree[ tree[dr].ls ].val - tree[ tree[dl].ls ].val;
	if(x >= k) return Query(tree[dl].ls, tree[dr].ls, l, mid, k);
	else return Query(tree[dl].rs, tree[dr].rs, mid + 1, r, k - x);
}

signed main()
{
	cin >> n >> m;
	for(int i = 1; i <= n; i++) cin >> a[i], b[++tot] = a[i];
	for(int i = 1; i <= m; i++) {
		char ch = getchar();
		if(ch == 'Q') cin >> q[i].l >> q[i].r >> q[i].k;
		else cin >> q[i].x >> q[i].y, b[++tot] = q[i].y;
	}
	Init();
	
	return 0;
}
\end{lstlisting}

\subsubsection{李超线段树}
\begin{lstlisting}
#include<bits/stdc++.h>
#define int long long
#define fi first
#define se second
#define pir make_pair
#define reg register
#define pdi pair<double, int>
using namespace std;
const int MAXN = 100005;
const int INF = 0x7fffffff;
const int MOD1 = 39989;
const int MOD2 = (int)1e9;

struct Tree {
	int l, r, id;
}tree[MAXN << 2];
struct Seg {
	double k, b;
	double f(int x) { return 1.0 * k * x + b; }
	Seg() {}
	Seg(int x0, int y0, int x1, int y1) {
		if(x0 == x1) k = 0, b = max(y0, y1);
		else {
			k = 1.0 * (y0 - y1) / (x0 - x1);
			b = y0 - 1.0 * k * x0;
		}
	}
}s[MAXN];
int n;

inline int ls(int x) { return x << 1; }
inline int rs(int x) { return x << 1 | 1; }
pdi Max(pdi u, pdi v) {
	if(u.fi > v.fi) return u;
	else if(u.fi < v.fi) return v;
	else {
		if(u.se < v.se) return u;
		else return v;
	}
}

void Build(int l, int r, int p) {
	tree[p].l = l;
	tree[p].r = r;
	tree[p].id = 0;
	if(l == r) return ;
	int mid = l + r >> 1;
	Build(l, mid, ls(p));
	Build(mid + 1, r, rs(p));
}

void Modify(int u, int nl, int nr, int p) {
	int l = tree[p].l, r = tree[p].r, v = tree[p].id, mid = l + r >> 1;
	double fu = s[u].f(mid), fv = s[v].f(mid);
	if(nl <= l && nr >= r) {
		if(l == r) {
			if(fu > fv) tree[p].id = u;
			return ;
		}
		if(s[u].k > s[v].k) {
			if(fu > fv) tree[p].id = u, Modify(v, nl, nr, ls(p));
			else Modify(u, nl, nr, rs(p));
		} else if(s[u].k < s[v].k) {
			if(fu > fv) tree[p].id = u, Modify(v, nl, nr, rs(p));
			else Modify(u, nl, nr, ls(p));			
		} else {
			if(s[u].b > s[v].b) tree[p].id = u;
		}
		return ;
	}
	if(nl <= mid) Modify(u, nl, nr, ls(p));
	if(nr > mid) Modify(u, nl, nr, rs(p));
}

pdi Query(int loc, int p) {
	int l = tree[p].l, r = tree[p].r, id = tree[p].id;
	if(l == r) return pir(s[id].f(loc), id);
	int mid = l + r >> 1;
	pdi u = pir(s[id].f(loc), id), v;
	if(loc <= mid) v = Query(loc, ls(p));
	else v = Query(loc, rs(p));
	return Max(u, v);
}

void Modify(int x0, int y0, int x1, int y1, int id) {
	s[id] = Seg(x0, y0, x1, y1);
	Modify(id, x0, x1, 1);
}

signed main()
{
	scanf("%lld", &n);
	Build(1, MOD1, 1);
	int last = 0, cnt = 0;
	for(int i = 1; i <= n; i++) {
		int op, k, x0, y0, x1, y1;
		scanf("%lld", &op);
		if(op == 0) {
			scanf("%lld", &k);
			k = (k + last - 1 + MOD1) % MOD1 + 1;
			last = Query(k, 1).se;
			printf("%lld\n", last);
		} else {
			scanf("%lld%lld%lld%lld", &x0, &y0, &x1, &y1);
			x0 = (x0 + last - 1 + MOD1) % MOD1 + 1;
			x1 = (x1 + last - 1 + MOD1) % MOD1 + 1;
			y0 = (y0 + last - 1 + MOD2) % MOD2 + 1;
			y1 = (y1 + last - 1 + MOD2) % MOD2 + 1;
			if(x0 > x1) swap(x0, x1), swap(y0, y1);
			Modify(x0, y0, x1, y1, ++cnt);
		}
	}
	return 0;
}
\end{lstlisting}

\subsubsection{可持久化线段树(标记永久化)}
\begin{lstlisting}

#include<bits/stdc++.h>
#define int long long
using namespace std;

const int MAXN = 100000 + 5;
const int MOD = 998244353;

struct node {
	int s, e, p;
} a[MAXN];

struct treenode {
	int ls, rs, sum = 0, cnt = 0;
} tree[MAXN * 50];

int n, m, tot = 0, root[MAXN];
map<int, int> Map;

int modify(int p, int nl, int nr, int v, int l = 1, int r = n) {
	int dir = ++tot;
	tree[dir] = tree[p];
	if(nl <= l && nr >= r) {
		tree[dir].sum += v;
		tree[dir].cnt += 1;
		return dir;
	}
	int mid = l + r >> 1;
	if(nl <= mid) tree[dir].ls = modify(tree[dir].ls, nl, nr, v, l, mid);
	if(nr > mid) tree[dir].rs = modify(tree[dir].rs, nl, nr, v, mid + 1, r);
	return dir;
}

pair<int, int> query(int p, int loc, int csum = 0, int ccnt = 0, int l = 1, int r = n) {
	if(l == r) return {csum + tree[p].sum, ccnt + tree[p].cnt};
	int mid = l + r >> 1;
	if(loc <= mid) return query(tree[p].ls, loc, csum + tree[p].sum, ccnt + tree[p].cnt, l, mid);
	else return query(tree[p].rs, loc, csum + tree[p].sum, ccnt + tree[p].cnt, mid + 1, r);
}

signed main() {
	ios::sync_with_stdio(false); cin.tie(0); cout.tie(0);
	cin >> m >> n;
	for(int i = 1; i <= m; i++) cin >> a[i].s >> a[i].e >> a[i].p, Map[ a[i].p ] = 1;
	sort(a + 1, a + 1 + m, [&](const node &a, const node &b) { return a.p < b.p; });
	for(int i = 1; i <= m; i++) root[i] = modify(root[i - 1], a[i].s, a[i].e, a[i].p);
	int pre = 1;
	for(int i = 1; i <= n; i++) {
		int x, a, b, c, k; cin >> x >> a >> b >> c;
		k = 1 + (a * pre % c + b) % c;
		int l = 1, r = m;
		while(l < r) {
			int mid = l + r - 1 >> 1;
			if(query(root[mid], x).second < k) l = mid + 1;
			else r = mid;
		}
		pre = query(root[l], x).first;
		cout << pre << "\n";
	}
	return 0;
}
\end{lstlisting}

\subsection{平衡树}

\subsubsection{普通平衡树}
您需要写一种数据结构,来维护一些数,其中需要提供以下操作:\\
\\
1. 插入 x 数\\
2. 删除 x 数(若有多个相同的数,因只删除一个)\\\
3. 查询 x 数的排名(排名定义为比当前数小的数的个数 +1 )\\
4. 查询排名为 x 的数\\
5. 求 x 的前驱(前驱定义为小于 x,且最大的数)\\
6. 求 x 的后继(后继定义为大于 x,且最小的数)\\
\\
input:\\
10\\
1 106465\\
4 1\\
1 317721\\
1 460929\\
1 644985\\
1 84185\\
1 89851\\
6 81968\\
1 492737\\
5 493598\\
\\
output:\\
106465\\
84185\\
492737\\

\subsubsection{文艺平衡树}
题目描述\\
这是一道模板题。\\
您需要写一种数据结构,来维护一个序列,其中需要提供以下操作:\\
翻转一个区间,例如原有序序列是 5 4 3 2 1,翻转区间是[2,4]的话,结果是 5 2 3 4 1。\\
\\
输入格式\\
第一行为 n,m , 表示初始序列有 n 个数,这个序列依次是{1,2,...n-1,n},m 表示翻转操作次数。\\
接下来 m 行每行两个数l,r\\
\\
输出格式\\
输出一行 n 个数字,表示原始序列经过 m 次变换后的结果。\\

\subsubsection{Treap普通平衡树}
\begin{lstlisting}
#include<bits/stdc++.h>

using namespace std;

const int SIZE = (int)1e5 + 5;
const int INF = 0x3f3f3f3f;

struct Treap {
	int l, r;	// 左右子节点 
	int val, dat;	// 关键码、权值 
	int cnt, size;	//  副本数、子树大小 
}a[SIZE];
int tot = 0, root, n;

int New(int val) {	// 创建新节点 
	a[++tot].val = val;
	a[tot].dat = rand();
	a[tot].cnt = a[tot].size = 1;
	return tot;
}

void Update(int p) {	// 更新子树大小 
	a[p].size = a[ a[p].l ].size + a[ a[p].r ].size + a[p].cnt;
}

void Build() {	// 建树 
	New(-INF), New(INF);
	root = 1, a[1].r = 2;
	Update(root);
}

void zig(int &p) {	// 右旋 
	int q = a[p].l;
	a[p].l = a[q].r, a[q].r = p, p = q;
	Update(a[p].r), Update(p);
}

void zag(int &p) {	// 左旋 
	int q = a[p].r;
	a[p].r = a[q].l, a[q].l = p, p = q;
	Update(a[p].l), Update(p);
}

void Insert(int &p, int val) {	// 插入val数
	if(p == 0) {
		p = New(val);
		return ;
	}
	if(val == a[p].val) {
		a[p].cnt++, Update(p);
		return ;
	}
	
	if(val < a[p].val) {
		Insert(a[p].l, val);
		if( a[p].dat < a[ a[p].l ].dat ) zig(p);	// 不满足大根堆性质 右旋 
	} else {
		Insert(a[p].r, val);
		if( a[p].dat < a[ a[p].r ].dat ) zag(p);	// 不满足大根堆性质 左旋		
	}
	Update(p);
}

void Remove(int &p, int val) {	// 删除val数 
	if(p == 0) return ;
	if(val == a[p].val) {
		if(a[p].cnt > 1) {
			a[p].cnt--, Update(p);
			return ;
		}
		if(a[p].l || a[p].r) {	// 不是叶子节点 
			if(a[p].r == 0 || a[ a[p].l ].dat > a[ a[p].r ].dat ) 
				zig(p), Remove(a[p].r, val);	// 右旋
			else 
				zag(p), Remove(a[p].l, val);	// 左旋 
			
			Update(p);
		} else p = 0;
		return ;
	}
	
	val < a[p].val ? Remove(a[p].l, val) : Remove(a[p].r, val);
	Update(p);
}

int GetRankByVal(int p, int val) {	// 查询val数的排名 
	if(p == 0) return 0;
	if(val == a[p].val) return a[ a[p].l ].size + 1;
	if(val < a[p].val) return GetRankByVal(a[p].l, val);
	else return GetRankByVal(a[p].r, val) + a[ a[p].l ].size + a[p].cnt;
}

int GetValByRank(int p, int rank) {	// 查询排名为rank的数 
	if(p == 0) return 0;
	if(a[ a[p].l ].size >= rank) return GetValByRank(a[p].l, rank);
	else if(a[ a[p].l ].size + a[p].cnt >= rank) return a[p].val;
	else return GetValByRank(a[p].r, rank - a[ a[p].l ].size - a[p].cnt);
}

int GetPre(int val) {	// 求val的前驱 
	int ans = 1;	// a[ans].val == -INF;
	int p = root;
	while(p) {
		if(val == a[p].val) {
			if(a[p].l > 0) {
				p = a[p].l;
				while(a[p].r > 0) p = a[p].r;	// 左子树中一直向右走
				ans = p; 
			}
			break;
		}
		if(a[p].val < val && a[p].val > a[ans].val) ans = p;	// 更新答案 
		
		val < a[p].val ? p = a[p].l : p = a[p].r;
	}
	return a[ans].val;
}

int GetNext(int val) {	// 求val的后继 
	int ans = 2;	// a[ans].val = INF;
	int p = root;
	while(p) {
		if(val == a[p].val) {
			if(a[p].r > 0) {
				p = a[p].r;
				while(a[p].l > 0) p = a[p].l;
				ans = p;
			}
			break;
		}
		if(a[p].val > val && a[p].val < a[ans].val) ans = p;
		
		val < a[p].val ? p = a[p].l : p = a[p].r;
	}
	return a[ans].val;
}

signed main()
{
	Build();
	srand((unsigned)time(NULL));
	scanf("%d", &n);
	while(n--) {
		int opt, x;
		scanf("%d%d", &opt, &x);
		switch(opt) {
			case 1 : Insert(root, x); break;
			case 2 : Remove(root, x); break;
			case 3 : printf("%d\n", GetRankByVal(root, x) - 1 ); break;	// 存在-INF 
			case 4 : printf("%d\n", GetValByRank(root, x + 1) ); break;	// 存在-INF 
			case 5 : printf("%d\n", GetPre(x) ); break;
			case 6 : printf("%d\n", GetNext(x) ); break;
		}
	}
	return 0;
}
\end{lstlisting}

\subsubsection{Splay普通平衡树}
\begin{lstlisting}
#include<bits/stdc++.h>

using namespace std;

const int SIZE = (int)1e5 + 5;
const int INF = 0x3f3f3f3f;

int son[SIZE][2], fa[SIZE];	// 左右子节点(son[0]代表左儿子,son[1]代表右儿子)、父亲节点
int val[SIZE];	// 权值 
int cnt[SIZE], size[SIZE];	// 副本数、子树大小 
int tot = 0, root = 0;	// 节点个数、根节点 
int n;

void Update(int p) {	// 更新子树大小 
	size[p] = size[ son[p][0] ] + size[ son[p][1] ] + cnt[p];
}

bool CheckRson(int p) {	// 判断节点p是不是右儿子 
	return p == son[ fa[p] ][1];
}

void Clear(int p) {	// 销毁节点p 
	son[p][0] = son[p][1] = fa[p] = val[p] = cnt[p] = size[p] = 0;
}

void Rotate(int p) {	//旋转p(根据p的儿子类型判断左旋还是右旋) 
	int f = fa[p], gf = fa[f];	// f->father gf->grandfather
	bool isRson = CheckRson(p);
	son[f][isRson] = son[p][isRson ^ 1];
	if(son[p][isRson ^ 1]) fa[ son[p][isRson ^ 1] ] = f;
	son[p][isRson ^ 1] = f;
	fa[f] = p, fa[p] = gf;
	if(gf) son[gf][ f == son[gf][1] ] = p;
	Update(p), Update(f);
}

void Splay(int p, int target = 0) { 	// 节点p旋转到节点target target=0表示根节点 
	for(int f = fa[p]; (f = fa[p]) != target, f; Rotate(p)) {
		if(fa[f] != target) Rotate( CheckRson(p) == CheckRson(f) ? f : p );
	}
	if(!target) root = p;
}

int GetPre() {	// 求x的前驱(将x插入 查询x左子树中最右边的节点 删除x)
	int cur = son[root][0];
	if(!cur) return cur;
	while(son[cur][1]) cur = son[cur][1];
	Splay(cur);
	return cur;
}

int GetNext() {	// 求x的前驱(将x插入 查询x右子树中最左边的节点 删除x)
	int cur = son[root][1];
	if(!cur) return cur;
	while(son[cur][0]) cur = son[cur][0];
	Splay(cur);
	return cur;
}

int GetRankByVal(int k) {	// 查询k数的排名 
	int res = 0, cur = root;
	while(1) {
		if(k < val[cur]) cur = son[cur][0];
		else {
			res += size[ son[cur][0] ];
			if(k == val[cur]) {
				Splay(cur);
				return res + 1;
			}
			res += cnt[cur];
			cur = son[cur][1];
		}
	}
}

int GetValByRank(int k) {	// 查询排名为k的数 
	int cur = root;
	while(1) {
		if(son[cur][0] && k <= size[ son[cur][0] ]) cur = son[cur][0];
		else {
			k -= size[ son[cur][0] ] + cnt[cur];
			if(k <= 0) {
				Splay(cur);
				return val[cur];
			}
			cur = son[cur][1];
		}
	}
}

void Insert(int k) {	// 插入k数
	if(!root) {
		val[++tot] = k, ++cnt[tot];
		root = tot, Update(root);
		return ;
	}
	int cur = root, f = 0;
	while(1) {
		if(val[cur] == k) {
			++cnt[cur];
			Update(cur), Update(f);
			Splay(cur);
			break;
		}
		f = cur;
		cur = son[cur][ val[cur] < k ];
		if(cur == 0) {
			val[++tot] = k, ++cnt[tot];
			fa[tot] = f, son[f][ val[f] < k ] = tot;
			Update(tot), Update(f);
			Splay(tot);
			break;
		}
	}
}

void Remove(int k) {	// 删除k数 
	GetRankByVal(k);
	if(cnt[root] > 1) {
		--cnt[root];
		Update(root);
		return ;
	}
	if(son[root][0] == 0 && son[root][1] == 0) {
		Clear(root);
		root = 0;
	} else if(son[root][0] == 0) {
		int cur = root;
		root = son[root][1];
		fa[root] = 0;
		Clear(cur);
	} else if(son[root][1] == 0) {
		int cur = root;
		root = son[root][0];
		fa[root] = 0;
		Clear(cur);
	} else {
		int cur = root, pre = GetPre();
		fa[ son[cur][1] ] = pre;
		son[pre][1] = son[cur][1];
		Clear(cur);
		Update(root);
	}
}

signed main()
{
	scanf("%d", &n);
	while(n--) {
		int opt, x;
		scanf("%d%d", &opt, &x);
		switch(opt) {
			case 1 : Insert(x); break;
			case 2 : Remove(x); break;
			case 3 : printf("%d\n", GetRankByVal(x) ); break;	// 存在-INF 
			case 4 : printf("%d\n", GetValByRank(x) ); break;	// 存在-INF 
			case 5 : Insert(x), printf("%d\n", val[GetPre()] ), Remove(x); break;
			case 6 : Insert(x), printf("%d\n", val[GetNext()] ), Remove(x); break;
		}
	}
	return 0;
}
\end{lstlisting}

\subsubsection{Splay文艺平衡树}
\begin{lstlisting}
#include<bits/stdc++.h>
using namespace std;
const int SIZE = (int)1e5 + 5;
const int INF = 0x3f3f3f3f;

int son[SIZE][2], fa[SIZE];	// 左右子节点(son[0]代表左儿子,son[1]代表右儿子)、父亲节点
int val[SIZE];	// 权值 
int cnt[SIZE], size[SIZE];	// 副本数、子树大小 
bool tag[SIZE]; 	// 标记 
int tot = 0, root = 0;	// 节点个数、根节点 
int n, m, a[SIZE];

void Update(int p) {	// 更新子树大小 
	size[p] = size[ son[p][0] ] + size[ son[p][1] ] + cnt[p];
}

bool CheckRson(int p) {	// 判断节点p是不是右儿子 
	return p == son[ fa[p] ][1];
}

void Clear(int p) {	// 销毁节点p 
	son[p][0] = son[p][1] = fa[p] = val[p] = cnt[p] = size[p] = 0;
}

void Rotate(int p) {	//旋转p(根据p的儿子类型判断左旋还是右旋) 
	int f = fa[p], gf = fa[f];	// f->father gf->grandfather
	bool isRson = CheckRson(p);
	son[f][isRson] = son[p][isRson ^ 1];
	if(son[p][isRson ^ 1]) fa[ son[p][isRson ^ 1] ] = f;
	son[p][isRson ^ 1] = f;
	fa[f] = p, fa[p] = gf;
	if(gf) son[gf][ f == son[gf][1] ] = p;
	Update(p), Update(f);
}

void Splay(int p, int target = 0) { 	// 节点p旋转到节点target的儿子下面 target=0表示根节点 
	for(int f; (f = fa[p]) != target; Rotate(p)) {
		if(fa[f] != target) Rotate( CheckRson(p) == CheckRson(f) ? f : p );
	}
	if(!target) root = p;
}

void PushDown(int p) {	// 下传反转标记 
	if(p && tag[p]) {
		int ls = son[p][0];
		int rs = son[p][1];
		tag[ls] ^= 1;
		tag[rs] ^= 1;
		swap(son[p][0], son[p][1]);
		tag[p] = 0;
	}
}

int FindNodeByRank(int k) {	// 查询值为k的节点 
	int cur = root;
	while(1) {
		PushDown(cur);
		if(son[cur][0] && k <= size[ son[cur][0] ]) cur = son[cur][0];
		else {
			k -= size[ son[cur][0] ] + 1;
			if(!k) return cur;
			else cur = son[cur][1];
		}
	}
}

int Build(int l, int r, int f) {	// 建树 
	if(l > r) return 0;
	int mid = l + r >> 1;
	int cur = ++tot;
	fa[cur] = f;
	cnt[cur] = 1;
	tag[cur] = 0;
	val[cur] = a[mid];	
	son[cur][0] = Build(l, mid - 1, cur);
	son[cur][1] = Build(mid + 1, r, cur);
	Update(cur);
	return cur;
}

void Traverse(int p) {	// 中序遍历 
	if(!p) return ;
	PushDown(p); 
	Traverse(son[p][0]);
	if(abs(val[p]) < INF) printf("%d ", val[p]);
	Traverse(son[p][1]);
}

void Reverse(int l, int r) {	// 反转区间[l,r] 
	int x = FindNodeByRank(l - 1 + 1);	// 存在-INF 
	int y = FindNodeByRank(r + 1 + 1);
	Splay(x, 0);
	Splay(y, x);
	PushDown(root);
	tag[ son[ son[root][1] ][0] ] ^= 1;
}

signed main()
{
	scanf("%d%d", &n, &m);
	a[1] = -INF, a[n + 2] = INF;
	for(int i = 1; i <= n; i++) a[i + 1] = i;
	root = Build(1, n + 2, 0);
	while(m--) {
		int l, r;
		scanf("%d%d", &l, &r);
		Reverse(l, r);
	}
	Traverse(root);
	return 0;
}
\end{lstlisting}

\subsubsection{FHQ-Treap}
\begin{lstlisting}
#include<bits/stdc++.h>
using namespace std;

const int MaxSize = (int)1e5 + 5;
const int INF = 0x3f3f3f3f;

// T为pair等类型时需要重载运算符
template <typename T, int MAXN>
class Treap {
private:
	struct Node {
		T val;
		int ls, rs, sze, priority;
	} tree[MaxSize];
	int seed, tot, root;
	int Top, Stack[MaxSize];
	
	int rand() { return seed = (int)(seed * 104831ll % 0x7fffffff); }
	
	void pushup(int p) {
		if(p) tree[p].sze = tree[ tree[p].ls ].sze + tree[ tree[p].rs ].sze + 1;
	} 
	
	int create(T val) {
		int p = Top ? Stack[Top--] : ++tot;
		tree[p].val = val;
		tree[p].sze = 1;
		tree[p].ls = tree[p].rs = 0;
		tree[p].priority = rand();
		return p;
	}
	
	// 将根为 p 的子树分裂成 x,y 两部分,x子树中全部小于等于val,y子树中全部大于val
	void split(int p, T val, int &x, int &y) {
		if(!p) return void(x = y = 0);
		if(val >= tree[p].val) {
			x = p;
			split(tree[p].rs, val, tree[p].rs, y);
		} else {
			y = p;
			split(tree[p].ls, val, x, tree[p].ls);
		}
		pushup(p);
	}
	/*
	// 将根为 p 的子树分裂成 x,y 两部分,x子树大小为sze
	void split(int p, int sze, int &x, int &y) {
		if(!p) return void(x = y = 0);
		if(tree[ tree[p].ls ].sze + 1 <= sze) {
			x = p;
			split(tree[p].rs, sze - (tree[ tree[p].ls ].sze + 1), tree[p].rs, y);
		} else {
			y = p;
			split(tree[p].ls, sze, x, tree[p].ls);
		}
		pushup(p);
	}
	*/
	int merge(int x, int y) {
		if(!x || !y) return x + y;
		if(tree[x].priority > tree[y].priority) {
			tree[x].rs = merge(tree[x].rs, y);
			pushup(x);
			return x;
		} else {
			tree[y].ls = merge(x, tree[y].ls);
			pushup(y);
			return y;
		}
	}
public:
	Treap() { seed = (int)(MAXN * 565463ll % 0x7fffffff); }
		
	void insert(T val) {
		int x, y;
		split(root, val - 1, x, y);
		root = merge( merge(x, create(val) ), y );
	}
	
	void remove(T val) {
		int x, y, z;
		split(root, val, x, z);
		split(x, val - 1, x, y);
		if(y) {
			Stack[++Top] = y;
			y = merge(tree[y].ls, tree[y].rs);
		}
		root = merge(merge(x, y), z);
	}
	
	int rank(T val) {
		int x, y, res;
		split(root, val - 1, x, y);
		res = tree[x].sze + 1;
		root = merge(x, y);
		return res;
	}
	
	int val(int rank) {
		int p = root;
		while(1) {
			if(tree[ tree[p].ls ].sze + 1 == rank) break;
			if(tree[ tree[p].ls ].sze + 1 > rank) p = tree[p].ls;
			else rank -= (tree[ tree[p].ls ].sze + 1), p = tree[p].rs;
		}
		return tree[p].val;
	}
	
	T prev(T val) {
		int x, y, p; T res;
		split(root, val - 1, x, y);
		p = x;
		while(tree[p].rs) p = tree[p].rs;
		res = tree[p].val;
		root = merge(x, y);
		return res;
	}
	
	T next(T val) {
		int x, y, p; T res;
		split(root, val, x, y);
		p = y;
		while(tree[p].ls) p = tree[p].ls;
		res = tree[p].val;
		root = merge(x, y);
		return res;
	}
	
	bool find(T val) {
		int x, y, z;
		split(root, val, x, z);
		split(x, val - 1, x, y);
		bool res = (tree[y].sze > 0);
		root = merge(merge(x, y), z);
		return res;
	}
	
	int size() { return tree[root].sze; }
}; Treap<int, 100005> fhqTreap;

signed main()
{
	int q, opt, x; scanf("%d", &q);
	while(q--) {
		scanf("%d%d", &opt, &x);
		switch(opt) {
			case 1 : fhqTreap.insert(x); break;
			case 2 : fhqTreap.remove(x); break;
			case 3 : printf("%d\n", fhqTreap.rank(x) ); break;
			case 4 : printf("%d\n", fhqTreap.val(x) ); break;
			case 5 : printf("%d\n", fhqTreap.prev(x) ); break;
			case 6 : printf("%d\n", fhqTreap.next(x) ); break;
		}
	}
	return 0;
}
\end{lstlisting}

\subsection{树套树}

\subsubsection{树状数组套主席树 求动态区间第k小}
\begin{lstlisting}
#pragma GCC optimize(2)
#include<bits/stdc++.h>
using namespace std;
const int MAXN = 100000 + 5;
const int INF = 0x7fffffff; 

struct Query {
	int l, r, k, x, y;
}q[MAXN];

map<int, int> Map;
vector<int> lRoot, rRoot;
int n, m, a[MAXN], cnt = 0, num[MAXN << 1], root[MAXN];
int tot, sum[MAXN << 7], ls[MAXN << 7], rs[MAXN << 7];

void Modify(int &rt, int l, int r, int loc, int val) {
	if(!rt) rt = ++tot;
	sum[rt] += val;
	if(l == r) return ;
	int mid = l + r >> 1;
	if(loc <= mid) Modify(ls[rt], l, mid, loc, val);
	else Modify(rs[rt], mid + 1, r, loc, val);
}

int QueryKth(int l, int r, int k) {
	if(l == r) return l;
	int mid = l + r >> 1, res = 0;
	for(auto i : rRoot) res += sum[ ls[i] ];
	for(auto i : lRoot) res -= sum[ ls[i] ];
	if(res >= k) {
		for(auto &i : lRoot) i = ls[i];
		for(auto &i : rRoot) i = ls[i];
		return QueryKth(l, mid, k);
	} else {
		for(auto &i : lRoot) i = rs[i];
		for(auto &i : rRoot) i = rs[i];		
		return QueryKth(mid + 1, r, k - res);
	}
}

void Modify(int pos, int val) {
	int loc = Map[ a[pos] ];
	for( ; pos <= n; pos += (pos & (-pos))) Modify(root[pos], 1, cnt, loc, val);
}

int Query(int l, int r, int k) {
	lRoot.clear(), rRoot.clear();
	for(int pos = l - 1; pos; pos -= (pos & (-pos))) lRoot.push_back(root[pos]);
	for(int pos = r; pos; pos -= (pos & (-pos))) rRoot.push_back(root[pos]);
	return QueryKth(1, cnt, k);
}

signed main()
{
	ios::sync_with_stdio(false); cin.tie(0);
	cin >> n >> m;
	for(int i = 1; i <= n; i++) cin >> a[i], Map[ a[i] ] = 1;
	for(int i = 1; i <= m; i++) {
		char ch; cin >> ch;
		if(ch == 'Q') cin >> q[i].l >> q[i].r >> q[i].k;
		else cin >> q[i].x >> q[i].y, Map[ q[i].y ] = 1;
	}
	for(auto &i : Map) i.second = ++cnt, num[cnt] = i.first;
	for(int i = 1; i <= n; i++) Modify(i, 1);
	for(int i = 1; i <= m; i++) {
		if(q[i].l) cout << num[ Query(q[i].l, q[i].r, q[i].k) ] << "\n";
		else {
			Modify(q[i].x, -1);
			a[ q[i].x ] = q[i].y;
			Modify(q[i].x, 1);
		}
	}
	return 0;
}
\end{lstlisting}

