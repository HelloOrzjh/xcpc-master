\section{字符串}

\subsection{字符串双哈希}
\begin{lstlisting}
#include<bits/stdc++.h>
using namespace std;

const int MAXN = 6e5 + 10;
const int INF = 0x3f3f3f3f;

namespace TwoHash {
	#define fi first
	#define se second
	#define pii pair<int, int>
	
	pii Pow[MAXN], Inv[MAXN];
	pii h[MAXN];
	int n;
	string S;
		
	const int BASE1 = 131;
	const int BASE2 = 129;
	const int MOD1 = 1e9 + 7;	
	const int MOD2 = 998244353;
	
	int qpow(int a, int p, int MOD) {
		a %= MOD, p %= MOD;
	    int res = 1;
	    while(p) {
	    	if(p & 1) res = 1ll * res * a % MOD;
	    	a = 1ll * a * a % MOD;
	    	p >>= 1;
		}
 	    return res;
	}
	
	pii Hash(int c, int p) { return {1ll * c * Pow[p].fi % MOD1, 1ll * c * Pow[p].se % MOD2}; }
	
	pii Add(pii x, pii y) {
		pii res = {0, 0};
		res.fi = 1ll * (x.fi + y.fi) % MOD1;
		res.se = 1ll * (x.se + y.se) % MOD2;
		return res;
	}
	
	pii Sub(pii x, pii y) {
		pii res = {0, 0};
		res.fi = 1ll * (x.fi - y.fi + MOD1) % MOD1;
		res.se = 1ll * (x.se - y.se + MOD2) % MOD2;
		return res;
	}
	
	pii Mul(pii x, pii y) {
		pii res = {0, 0};
		res.fi = 1ll * x.fi * y.fi % MOD1;
		res.se = 1ll * x.se * y.se % MOD2;
		return res;
	}
	
	pii Div(pii x, pii y) {
		pii res = {0, 0};
		res.fi = 1ll * x.fi * qpow(y.fi, MOD1 - 2, MOD1) % MOD1;
		res.se = 1ll * x.se * qpow(y.se, MOD1 - 2, MOD1) % MOD1;
		return res;
	}
	
	pii HashVal(int l, int r) { return Mul( Sub(h[r], h[l - 1]), Inv[l - 1]); }
	
	void Init() {
		Pow[0] = Inv[0] = {1, 1}; Inv[1] = {qpow(BASE1, MOD1 - 2, MOD1), qpow(BASE2, MOD2 - 2, MOD2)};
		for(int i = 1; i <= MAXN - 5; i++) Pow[i] = Mul(Pow[i - 1], {BASE1, BASE2});
		for(int i = 2; i <= MAXN - 5; i++) Inv[i] = Mul(Inv[i - 1], Inv[1]);	
	}
	
	void Build(string T) {
		n = T.length(); S = " " + T;
		h[0] = {0, 0};
		for(int i = 1; i <= n; i++) h[i] = Add(h[i - 1], Hash(S[i], i) );
	}
}

signed main() 
{
	TwoHash::Init();
	set<pii> Set;
	int T; cin >> T;
	while(T--) {
		string S; cin >> S;
		TwoHash::Build(S);
		Set.insert( TwoHash::h[S.length()] );
	}
	cout << Set.size();
	return 0;
}

\end{lstlisting}

\subsection{Trie}
\begin{lstlisting}
int Hash(char ch) { return ch - 'a' + 1; }

void Insert(string S) {
	int u = 0;
	for(auto i : S) {
		if(!trie[u][Hash(i)]) trie[u][Hash(i)] = ++cnt;
		u = trie[u][Hash(i)];
	}
	sum[u]++;
}
\end{lstlisting}

\subsection{KMP}
\begin{lstlisting}
#include<bits/stdc++.h>

using namespace std;

const int MAXN = 1e6 + 5;

char S1[MAXN], S2[MAXN];
int l1, l2, pmt[MAXN], p;	//pmt数组(部分匹配表) 向右偏移一位为next数组 并让next[0] = -1

int main()
{
	cin >> S1 + 1 >> S2 + 1;
	l1 = strlen(S1 + 1), l2 = strlen(S2 + 1);
	p = pmt[0] = 0;
	for(int i = 2; i <= l2; i++) {
		while(p && S2[p + 1] != S2[i]) p = pmt[p];
		if(S2[p + 1] == S2[i]) p++;
		pmt[i] = p;
	}
	p = 0;
	for(int i = 1; i <= l1; i++) {
		while(p && S2[p + 1] != S1[i]) p = pmt[p];
		if(S2[p + 1] == S1[i]) p++;
		if(p == l2) {
			printf("%d\n", i - p + 1);
			p = pmt[p];
		}
	}
	for(int i = 1; i <= l2; i++) printf("%d ", pmt[i]);
	return 0;
}
\end{lstlisting}

\subsection{exKMP}
\begin{lstlisting}
#include<bits/stdc++.h>
using namespace std;
const int MAXN = (int)2e7 + 5;

char S[MAXN], T[MAXN];
int z[MAXN], lcp[MAXN];

void Z_Function(char* T) {	// z[i] = lcp(s[i ... n-1], s)
	int n = strlen(T);
	z[0] = n;
	int l = 0, r = 0;
	
	for(int i = 1; i < n; i++) {
		if(i <= r) z[i] = min(z[i - l], r - i + 1);
		while(i + z[i] < n && T[z[i]] == T[i + z[i]]) z[i]++;
		if(i + z[i] - 1 > r) l = i, r = i + z[i] - 1;
	}
	
	//for(int i = 0; i < n; i++) cout << z[i] << " "; cout << endl;
}

void exKMP(char* S, char* T) {
	int sLen = strlen(S);
	int tLen = strlen(T);
	Z_Function(T);
	
	int p = 0;
	while(S[p] == T[p] && p < min(sLen, tLen)) p++;
	lcp[0] = p;
	
	int l = 0, r = 0;
	for(int i = 1; i < sLen; i++) {
		if(i <= r) lcp[i] = min(z[i - l], r - i + 1);
		while(i + lcp[i] < sLen && lcp[i] < tLen && S[i + lcp[i]] == T[lcp[i]]) lcp[i]++;
		if(i + lcp[i] - 1 > r) l = i, r = i + lcp[i] - 1;
	}
	
	//for(int i = 0; i < sLen; i++) cout << lcp[i] << " "; cout << endl;
}

signed main()
{
	scanf("%s%s", S, T);
	exKMP(S, T);
	return 0;
}

/*
input:
aaaabaa
aaaaa

z function : {5 4 3 2 1}
lcp function : {4 3 2 1 0 2 1} 
*/
\end{lstlisting}

\subsection{Manacher}
\begin{lstlisting}
/*
Manacher算法:
先在两个字符串中插入某个字符('$') 避免分别处理奇回文和偶回文的情况
设置两个指针maxR(前i个字符能回文扩展到的最右端) pos(前i个字符中哪个字符能回文扩展到最右端)

每次扫描到第i个字符时
①如果i<maxR,更新f[i] ②暴力拓展maxR(maxR从起点到终点且不会往回退) ③maxR增大时更新maxR和pos
*/

#include<bits/stdc++.h>
using namespace std;

const int MAXN = (int)1.1e7 + 5;
char S[MAXN << 1], T[MAXN << 1];
int f[MAXN << 1], n;

void Manacher() {
	T[0] = '#'; T[1] = '$';
	for(int i = 1; i <= n; i++) T[i * 2] = S[i], T[i * 2 + 1] = '$';
	n = n * 2 + 1;
	for(int i = 0; i <= n; i++) S[i] = T[i];
	
	int maxR = 0, pos = 0;
	for(int i = 1; i <= n; i++) {
		if(i < maxR) f[i] = min(f[pos * 2 - i], maxR - i);
		while(i - f[i] - 1 > 0 && i + f[i] + 1 <= n && S[i + f[i] + 1] == S[i - f[i] - 1]) f[i]++;
		if(i + f[i] > maxR) maxR = i + f[i], pos = i;
	}
}

int main()
{
	scanf("%s", S + 1); n = strlen(S + 1);
	Manacher();
	int ans = 0;
	for(int i = 1; i <= n; i++) ans = max(ans, f[i]);
	printf("%d", ans);
	return 0;
}
\end{lstlisting}

\subsection{最小最大表示法}
\begin{lstlisting}
/*
S的最小表示:与S循环同构的所有字符串中字典序最小的字符串(最大表示同理) 
*/
int getMin(string S) {
	int n = S.length(), i = 0, j = 1, k = 0;
	while(i < n && j < n && k < n) {
		if(S[(i + k) % n] == S[(j + k) % n]) k++;
		else {
			if(S[(i + k) % n] > S[(j + k) % n]) i = i + k + 1;
			else j = j + k + 1;
			
			if(i == j) i++;
			k = 0;
		}
	}
	return min(i, j);
}

int getMax(string S) {
  	int n = S.length(), i = 0, j = 1, k = 0;
	while(i < n && j < n && k < n) {
		if(S[(i + k) % n] == S[(j + k) % n]) k++;
		else {
			if(S[(i + k) % n] < S[(j + k) % n]) i = i + k + 1;
			else j = j + k + 1;
			
			if(i == j) i++;
			k = 0;
		}
	}
	return min(i, j);
}
\end{lstlisting}

\subsection{AC自动机}
\begin{lstlisting}
#include<bits/stdc++.h>
using namespace std;
const int MAXN = (int)1e6 + 5;

char S[MAXN], T[MAXN];
int n, cnt, trie[MAXN][30], fail[MAXN * 30], num[MAXN * 30];
queue<int> Q;

inline int Hash(char x) { return x - 'a' + 1; }

void Trie(char* S) {
	int cur = 1;
	int len = strlen(S);
	for(int i = 0; i < len; i++) {
		int x = Hash(S[i]);
		if(!trie[cur][x]) trie[cur][x] = ++cnt;
		cur = trie[cur][x];
	}
	num[cur]++;
}

void GetFail() {
	for(int i = 1; i <= 26; i++) trie[0][i] = 1;
	Q.push(1);
	fail[1] = 0;
	while(!Q.empty()) {
		int u = Q.front();
		Q.pop();
		int faFail = fail[u];
		for(int i = 1; i <= 26; i++) {
			int v = trie[u][i];
			if(v) fail[v] = trie[faFail][i], Q.push(v);
			else trie[u][i] = trie[faFail][i];
		}
	}
}

int main()
{
	scanf("%d", &n);
	cnt = 1;
	for(int i = 1; i <= n; i++) {
		scanf("%s", S);
		Trie(S);
	}
	GetFail(); 
	scanf("%s", T);
	
	int cur = 1, ans = 0, len = strlen(T);
	for(int i = 0; i < len; i++) {
		cur = trie[cur][Hash(T[i])];
		for(int t = cur; t && ~num[t]; t = fail[t]) ans += num[t], num[t] = -1;
	}
	printf("%d\n", ans);
	return 0;
}
\end{lstlisting}

\subsection{后缀数组}
\begin{lstlisting}
/* 注意常数 */
//#pragma GCC optimize(2)
#include<bits/stdc++.h>
using namespace std;

const int MAXN = 1e6 + 10;

char S[MAXN];
int n, m, sa[MAXN], rk[MAXN], oldrk[MAXN << 1], id[MAXN], px[MAXN], cnt[MAXN]; 

bool cmp(int x, int y, int w) {
	return oldrk[x] == oldrk[y] && oldrk[x + w] == oldrk[y + w];
}

void SuffixArray(char *s) {
	// getSA	
	n = strlen(s + 1);
	int w, p, i, m = 300, k;
	for(i = 1; i <= n; ++i) ++cnt[ rk[i] = s[i] ];
	for(i = 1; i <= m; ++i) cnt[i] += cnt[i - 1];
	for(i = n; i >= 1; --i) sa[ cnt[ rk[i] ]-- ] = i;
	
	for(w = 1; ; w <<= 1, m = p) {
		for(p = 0, i = n; i > n - w; --i) id[++p] = i;
		for(i = 1; i <= n; ++i) if(sa[i] > w) id[++p] = sa[i] - w;
		for(i = 0; i <= m; ++i) cnt[i] = 0;
		for(i = 1; i <= n; ++i) ++cnt[ px[i] = rk[ id[i] ] ];
		for(i = 1; i <= m; ++i) cnt[i] += cnt[i - 1];
		for(i = n; i >= 1; --i) sa[ cnt[ px[i] ]-- ] = id[i];
		for(i = 1; i <= n; ++i) oldrk[i] = rk[i];
		for(p = 0, i = 1; i <= n; ++i) rk[ sa[i] ] = cmp(sa[i], sa[i - 1], w) ? p : ++p;
	
		if(p == n) {
			for(int i = 1; i <= n; ++i) sa[ rk[i] ] = i;	
			break;
		}
	}
	//for(int i = 1; i <= n; ++i) cout << sa[i] << " "; cout << "\n";
	
	//getHeight
	for(i = 1, k = 0; i <= n; ++i) {
		if(rk[i] == 0) continue;
		if(k) --k;
		while(S[i + k] == S[ sa[ rk[i] - 1 ] + k ]) ++k;
		height[ rk[i] ] = k;
	}
}

signed main()
{
	ios::sync_with_stdio(false); cin.tie(0); cout.tie(0);
	cin >> S + 1;
	SuffixArray(S);
	return 0;
}

\end{lstlisting}

\subsection{后缀自动机}
\begin{lstlisting}
#include<bits/stdc++.h>
#pragma GCC optimize("-Ofast")
using namespace std;
const int MAXN = (int)4e5 + 5;

struct state {
	int len, link, size;
	int next[30];
}st[MAXN << 1];

int sz, last, n, p, q;
long long f[MAXN];
char s[MAXN];

void init() {
	for(int i = 0; i < sz; i++) {
		for(int j = 0; j < 26; j++) st[i].next[j] = 0;
		st[i].len = st[i].link = st[i].size = 0;
	}
	sz = 0;
	
	st[0].len = 0;
	st[0].link = -1;
	sz++;
	last = 0; 
}

void extend(int c) {
	int cur = sz++;
	st[cur].size = 1;
	st[cur].len = st[last].len + 1;
	int p = last;
	while(p != -1 && !st[p].next[c]) {
		st[p].next[c] = cur;
		p = st[p].link;
	} 
	if(p == -1) st[cur].link = 0;
	else {
		int q = st[p].next[c];
		if(st[p].len + 1 == st[q].len) st[cur].link = q;
		else {
			int clone = sz++;
			st[clone].len = st[p].len + 1;
			for(int i = 0; i < 26; i++) st[clone].next[i] = st[q].next[i];
			st[clone].link = st[q].link;
			while(p != -1 && st[p].next[c] == q) {
				st[p].next[c] = clone;
				p = st[p].link;
			}
			st[q].link = st[cur].link = clone;
		}
	}
	last = cur;
}

\end{lstlisting}