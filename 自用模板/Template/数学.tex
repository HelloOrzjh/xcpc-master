\section{数学}
\subsection{快速幂}
\begin{lstlisting}
// 快速乘 适用正数 
int qmul(int a, int b, int MOD) {
	int ans = 0;
	while(b > 0) {
		if(b & 1) ans = (ans + a) % MOD;
		b >>= 1;
		a = (a << 1) % MOD;
	}
	return ans;
}

// 快速幂 
int qpow(int a, int p, int MOD) {
	int ans = 1;
	while(p > 0) {
		if(p & 1) ans = ans * a % MOD;
		p >>= 1;
		a = a * a % MOD;
	}
	return ans;
}

// 矩阵快速幂 
struct Martix {
	long long c[MAXN][MAXN];
}A, E;

Martix operator * (const Martix &a, const Martix &b) {
	Martix Ans;
	for(int i = 1; i <= n; i++) for(int j = 1; j <= n; j++) Ans.c[i][j] = 0;
	for(int k = 1; k <= n; k++) {
		for(int i = 1; i <= n; i++) {
			for(int j = 1; j <= n; j++) Ans.c[i][j] = (Ans.c[i][j] + a.c[i][k] * b.c[k][j] % MOD) % MOD;
		}
	}
	return Ans;
}

Martix MartixQuickPow(Martix A, int k) {
	Martix ans, now;
	for(int i = 1; i <= n; i++) {
		for(int j = 1; j <= n; j++) ans.c[i][j] = 0;
		ans.c[i][i] = 1;
	}
	for(int i = 1; i <= n; i++) for(int j = 1; j <= n; j++) now.c[i][j] = A.c[i][j];
	while(k > 0) {
		if(1 & k) ans = ans * now;
		now = now * now; 
		k = k >> 1;
	}
	return ans;
}
\end{lstlisting}

\subsection{欧拉函数}
\begin{lstlisting}
int phi(int n) {
    int res = n;
    for(int i = 2; i * i <= n; i++) {
        if(n % i == 0) res = res / i * (i - 1);
        while(n % i == 0) n /= i;
    }
    if(n > 1) res = res / n * (n - 1);
    return res;
}

int phi[MAXN];
void init(int n) {
    // 复杂度 O(nloglogn)
    for(int i = 1; i <= n; i++) phi[i] = i; // 除1外没有数的欧拉函数是本身,所以如果phi[i] = i则说明未被筛到
    for(int i = 2; i <= n; i++) {
        if(phi[i] == i) { // 未被筛到
            for(int j = i; j <= n; j += i) phi[j] = phi[j] / i * (i - 1); // 所有含有该因子的数都进行一次操作
        }
    }

    // 复杂度 O(n)
    phi[1] = 1;
    for(int i = 2; i <= n; i++) {
        if(!isnp[i]) primes.push_back(i), phi[i] = i - 1;
        for(int p : primes) {
            if(p * i > n) break;
            isnp[p * i] = 1;
            if(i % p == 0) {
                phi[p * i] = phi[i] * p;
                break;
            } else {
                phi[p * i] = phi[p] * phi[i];
            }
        }
    }
}

int EulerPow(int a, string b, int MOD) {
    int phiMOD = phi(MOD), power = 0, flag = 0;
    for(auto i : b) {
        power = power * 10 + i - '0';
        if(power >= phiMOD) flag = 1;
        power %= phiMOD;
    }
    if(flag) power += phiMOD;
    return qpow(a, power, MOD);
}
\end{lstlisting}

\subsection{线性筛}
\begin{lstlisting}
#include<bits/stdc++.h>

using namespace std;

const int MAXN = 2e7 + 5;

int n, m, isprime[MAXN], prime[MAXN], cnt;

void Init(int n = 2e7) {
	cnt = 0;
	isprime[1] = 1;
	for(int i = 2; i <= n; i++) {
		if(!isprime[i]) prime[++cnt] = i;
		for(int j = 1; j <= cnt && i * prime[j] <= n; j++) {
			isprime[i * prime[j]] = 1;
			if(i % prime[j] == 0) break;
		}
	}
	//for(int i = 1; i <= cnt; i++) cout << prime[i] << " "; cout << endl;
}

int main()
{
	scanf("%d", &n);
	Init(n);
	return 0;
}
\end{lstlisting}

\subsection{整除分块}
\begin{lstlisting}
for(int l = 1, r; l <= n; l = r + 1) r = n / (n / l);	// [l, r]为当前整除分块 区间内每个 n / i 相同 
\end{lstlisting}

\subsection{组合数处理}
\begin{lstlisting}
int inv(int x, int MOD) {	// 求逆元 
	return qpow(x, MOD - 2, MOD);
}

void Init() {
	fact[0] = 1;
	for(int i = 1; i <= n; i++) fact[i] = fact[i - 1] * i % MOD;

	inv[n] = inv(fact[n], MOD);
	for(int i = n - 1; i >= 0; i--) inv[i] = inv[i + 1] * (i + 1) % MOD;
}

int C(int n, int m) {
	if(m > n) return 0;
	return fact[n] * inv[m] % MOD * inv[n - m] % MOD;
}

int Lucas(int n, int m) {
	if(m == 0) return 1;
	return C(n % MOD, m % MOD) * Lucas(n / MOD, m / MOD) % MOD;
}
\end{lstlisting}