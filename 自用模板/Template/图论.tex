\section{图论}

\subsection{DFS序}
\begin{lstlisting}
/*
对于节点u 其子树范围 [ dfn[u], dfn[u] + sze[u] - 1 ] 
可用线段树等数据结构维护 
*/
void DFS(int u, int fa) {
	dfn[u] = ++cnt;
	sze[u] = 1;
	for(auto v : G[u]) {
		if(v == fa) continue;
		DFS(v, u);
		sze[u] += sze[v];
	}
}
\end{lstlisting}

\subsection{LCA}
\begin{lstlisting}
int n, m, dep[MAXN], vis[MAXN], lca[MAXN][25];
vector<int> G[MAXN];

void DFS(int u, int ftr) {
	dep[u] = dep[ftr] + 1;
	vis[u] = 1;
	lca[u][0] = ftr;
	for(int i = 1; i <= 20; i++) {
		if(dep[u] < (1 << i)) break;
		lca[u][i] = lca[lca[u][i - 1]][i - 1];
	}
	for(int i = 0; i < G[u].size(); i++) {
		int v = G[u][i];
		if(vis[v]) continue;
		DFS(v, u);
	}
}

int LCA(int x, int y) {
	int u = x, v = y;
	if(dep[u] > dep[v]) swap(u, v); //dep[u] <= dep[v]
	for(int i = 20; i >= 0; i--) {
		if((1 << i) & (dep[v] - dep[u])) v = lca[v][i];
	}
	for(int i = 20; i >= 0; i--) {
		if(lca[u][i] != lca[v][i]) {
			u = lca[u][i];
			v = lca[v][i];
		}
	}
	return u == v ? u : lca[u][0];
}
\end{lstlisting}

\subsection{树的重心}
\begin{lstlisting}
int size[MAXN], weight[MAXN], centroid[3];  
 
void GetCentroid(int cur, int fa) {
	size[cur] = 1;
	weight[cur] = 0;
	for(int i = head[cur]; i != -1; i = e[i].nxt) {
		if(e[i].to == fa) continue;
    	GetCentroid(e[i].to, cur);
    	size[cur] += size[ e[i].to ];
    	weight[cur] = max(weight[cur], size[ e[i].to ]);
    }
	weight[cur] = max(weight[cur], n - size[cur]);
	if(weight[cur] <= n / 2) centroid[ ++centroid[0] ] = cur;
}
\end{lstlisting}

\subsection{树的直径}
\begin{lstlisting}
int n, d1[MAXN], d2[MAXN], d = 0;
vector<int> G[MAXN];

void DFS(int u, int fa) {
	d1[u] = d2[u] = 0;
	for(auto v : G[u]) {
		if(v == fa) continue;
		DFS(v, u);
		int cur = d1[v] + 1;
		if(cur > d1[u]) d2[u] = d1[u], d1[u] = cur;
		else if(cur > d2[u]) d2[u] = cur;
	} 
	d = max(d, d1[u] + d2[u]);
}
\end{lstlisting}

\subsection{最小生成树}
\begin{lstlisting}
int Prim() {
	memset(dist, 0x3f, sizeof(dist));
	int res = 0;
	for(int i = 0; i < n; i++) {
		int t = -1;
		for(int j = 1; j <= n; j++) {
			if(!st[j] && (t == -1 || dist[t] > dist[j])) t = j;
		}
		if(i && dist[t] == INF) return INF;	// 无解 
		
		if(i) res += dist[t];
		st[t] = true;
		
		for(int j = 1; j <= n; j++) dist[j] = min(dist[j], G[t][j]);
	}
	return res;
}

int Kruskal() {
	for(int i = 1; i <= n; i++) f[i] = i;
	sort(e + 1, e + 1 + m, cmp); // 按边权排序
	ans = 0;
	for(int i = 1; i <= m; i++) {
		int fu = find(e[i].u), fv = find(e[i].v);
		if(fu == fv) continue;
		if(fu > fv) swap(fu, fv);
		f[fv] = fu;
		ans += e[i].w;
		e[i].flag = 1;
		G[ e[i].u ].push_back( make_pair(e[i].v, e[i].w) );
		G[ e[i].v ].push_back( make_pair(e[i].u, e[i].w) );
	}
	return ans;
}
\end{lstlisting}

\subsection{严格次小生成树}
\begin{lstlisting}
#include<bits/stdc++.h>
#define int long long
using namespace std;

const int MAXN = (int)1e5 + 5;
const int MAXM = (int)3e5 + 5;
struct Node {
	int u, v, w, flag;
}e[MAXM];
int lca[MAXN][30], maxx1[MAXN][30], maxx2[MAXN][30], dep[MAXN], n, m, f[MAXN], ans, vis[MAXN];
vector< pair<int, int> > G[MAXN];

bool cmp(Node &a, Node &b) { return a.w < b.w; }
int find(int x) { return ( x == f[x] ? x : ( f[x] = find(f[x]) ) ); }

void Kruskal() {
	for(int i = 1; i <= n; i++) f[i] = i;
	sort(e + 1, e + 1 + m, cmp);
	
	ans = 0;
	for(int i = 1; i <= m; i++) {
		int fu = find(e[i].u), fv = find(e[i].v);
		if(fu == fv) continue;
		if(fu > fv) swap(fu, fv);
		f[fv] = fu;
		ans += e[i].w;
		e[i].flag = 1;
		G[ e[i].u ].push_back( make_pair(e[i].v, e[i].w) ), G[ e[i].v ].push_back( make_pair(e[i].u, e[i].w) );
	}
}

void DFS(int u, int ftr) {
	// maxx1维护最大值 maxx2维护次大值 
	dep[u] = dep[ftr] + 1;
	vis[u] = 1;
	for(int i = 1; i <= 20; i++) {
		lca[u][i] = lca[ lca[u][i - 1] ][i - 1];
		maxx1[u][i] = max( maxx1[u][i - 1], maxx1[ lca[u][i - 1] ][i - 1] ); 
		if(maxx1[u][i - 1] == maxx1[ lca[u][i - 1] ][i - 1]) 
			maxx2[u][i] = max( maxx2[u][i - 1], maxx2[ lca[u][i - 1] ][i - 1] );
		else 
			maxx2[u][i] = max( min( maxx1[u][i - 1], maxx1[ lca[u][i - 1] ][i - 1] ), max( maxx2[u][i - 1], maxx2[ lca[u][i - 1] ][i - 1] ) );
	}
	
	for(auto i : G[u]) {
		int v = i.first, w = i.second;
		if(vis[v]) continue;
		lca[v][0] = u; maxx1[v][0] = w; maxx2[v][0] = 0;
		DFS(v, u);
	}
}

int LCA(int u, int v) {
	if(dep[u] < dep[v]) swap(u, v);
	for(int i = 20; i >= 0; i--) {
		if((1 << i) & (dep[u] - dep[v])) u = lca[u][i];
	}
	for(int i = 20; i >= 0; i--) {
		if(lca[u][i] != lca[v][i]) {
			u = lca[u][i];
			v = lca[v][i];
		}
	}
	return u == v ? u : lca[u][0];
}

int Work(int u, int l, int w) {
	int max1 = 0, max2 = 0;
	for(int i = 20; i >= 0; i--) {
		if((1 << i) & (dep[u] - dep[l])) {
			max1 = max(max1, maxx1[u][i]);
			max2 = max(max2, maxx2[u][i]);
			u = lca[u][i]; 
		}
	}
	if(w - max1 == 0) return w - max2;
	return w - max1;
}

signed main()
{
	scanf("%lld%lld", &n, &m);
	for(int i = 1; i <= m; i++) {
		int u, v, w;
		scanf("%lld%lld%lld", &u, &v, &w);
		e[i].u = u, e[i].v = v, e[i].w = w, e[i].flag = 0;
	}
	Kruskal();
	DFS(1, 0);
	
	int d = 1e14;
	for(int i = 1; i <= m; i++) {
		if(e[i].flag) continue;
		int u = e[i].u, v = e[i].v, w = e[i].w;
		int l = LCA(u, v);
		d = min(d, Work(u, l, w) );
		d = min(d, Work(v, l, w) );
	}
	printf("%lld", ans + d);
	return 0;
}
\end{lstlisting}

\subsection{Dijkstra}
\begin{lstlisting}
#include<bits/stdc++.h>
using namespace std;

const int INF = 0x3f3f3f3f;
const int MAXN = (int)1e5 + 5;
const int MAXM = (int)2e5 + 5;
struct Node {
	int u, w;
	bool operator < (const Node &a) const { return w > a.w; } 
};
struct Edge {
	int v, w, nxt;
}e[MAXM];
int n, m, s, cnt = 0, head[MAXN], vis[MAXN], dist[MAXN];
priority_queue<Node> Q;

void Add(int u, int v, int w) {
	e[++cnt].v = v;
	e[cnt].w = w;
	e[cnt].nxt = head[u];
	head[u] = cnt;
}

void Dijkstra(int s) {
	for(int i = 1; i <= n; i++) vis[i] = 0, dist[i] = INF;
	dist[s] = 0;
	Q.push((Node){s, 0});
	while(!Q.empty()) {
		Node now = Q.top();
		int u = now.u, w = now.w;
		Q.pop();
		if(vis[u]) continue;
		vis[u] = 1;
		for(int i = head[u]; i != -1; i = e[i].nxt) {
			if(dist[e[i].v] > dist[u] + e[i].w) {
				dist[e[i].v] = dist[u] + e[i].w;
				Q.push((Node){e[i].v, dist[e[i].v]});
			}
		}
	}
}

int main(){
	scanf("%d%d%d", &n, &m, &s);
	memset(head, -1, sizeof(head));
	for(int i = 1; i <= m; i++) {
		int u, v, w;
		scanf("%d%d%d", &u, &v, &w);
		Add(u, v, w);
	}
	Dijkstra(s);
	for(int i = 1; i <= n; i++) printf("%d ", dist[i]);
	return 0;
}
\end{lstlisting}

\subsection{SPFA}
\begin{lstlisting}
int n;	// 总点数 
int h[N], w[N], e[N], ne[N], idx;	// 邻接表存边 
int dist[N];	// 存储每个点到1号点的最短距离 
bool st[N];	// 存储每个点是否在队列中 

// 求1号点到n号点的最短距离,如果无法到达返回-1 
int SPFA() {
	memset(dist, 0x3f, sizeof dist);
	queue<int> q;
	q.push(1);
	dist[1] = 0, st[1] = true;
	while(!q.empty()) {
		auto t = q.front(); q.pop();
		st[t] = false;
		for(int i = h[t]; i != -1; i = ne[i]) {
			int j = e[i];
			if(dist[j] > dist[t] + w[i]) {
				dist[j] = dist[t] + w[i];
				if(!st[j]) q.push(j), st[j] = true;
			}
		}
	}
	if(dist[n] >= 0x3f3f3f3f) return -1;
	return dist[n];
}
\end{lstlisting}

\subsection{SPFA负环}
\begin{lstlisting}
int n; // 总点数
int h[N], w[N], e[N], ne[N], idx; // 邻接表存储所有边
int dist[N], cnt[N]; // dist[x]存储1号点到x的最短距离,cnt[x]存储1到x的最短路中经过的点数
bool st[N]; // 存储每个点是否在队列中

// 如果存在负环,则返回true,否则返回false。
bool SPFA() {
	// 不需要初始化dist数组
	// 原理:如果某条最短路径上有n个点(除了自己),那么加上自己之后一共有n+1个点,由抽屉原理一定有两个点相同,所以存在环。
	
	queue<int> q;
	for (int i = 1; i <= n; i++) q.push(i), st[i] = true;
	
	while(!q.empty()) {
		auto t = q.front(); q.pop();
		st[t] = false;
		for(int i = h[t]; i != -1; i = ne[i]) {
			int j = e[i];
			if(dist[j] > dist[t] + w[i]) {
				dist[j] = dist[t] + w[i];
				cnt[j] = cnt[t] + 1;
				if(cnt[j] >= n) return true; // 如果从1号点到x的最短路中包含至少n个点(不包括自己),则说明存在环
				if(!st[j]) q.push(j), st[j] = true;
			}
		}
	}
	return false;
}
\end{lstlisting}

\subsection{Floyd}
\begin{lstlisting}
void Solve1() {
	for(int i = 1; i <= n; i++) {
		for(int j = 1; j <= n; j++) {
			if(i == j) dp[i][j] = 0;
			else dp[i][j] = INF;
		}
	}
	for(int k = 1; k <= n; k++) {
		for(int i = 1; i <= n; i++) {
			for(int j = 1; j <= n; j++) {
				dp[i][j] = min(dp[i][j], dp[i][k] + dp[k][j]);
			}
		}
	}
}

/*
已知一个有向图中任意两点之间是否有连边,要求判断任意两点是否连通。
Floyd实现传递闭包 bitset优化 
*/
void Solve2()
{
	for(int k = 1; k <= n; k++) {
		for(int i = 1; i <= n; i++) {
			if(G[i][k]) G[i] |= G[k];
		}
	} 
	for(int k = 1; k <= n; k++) {
		for(int i = 1; i <= n; i++) {
			for(int j = 1; j <= n; j++) {
				G[i][j] |= G[i][k] & G[k][j];
			}
		}
	}
}
\end{lstlisting}

\subsection{差分约束}
\begin{lstlisting}
#include<bits/stdc++.h>
using namespace std;

const int MAXN = 5005;
const int INF = 0x3f3f3f3f;

struct Node { int v, w; };

vector<Node> e[MAXN];
int head[MAXN], vis[MAXN], cnt[MAXN], dis[MAXN], n, m, ecnt;
queue<int> Q;

bool SPFA(int s) {
	for(int i = 0; i <= n; i++) dis[i] = INF, vis[i] = cnt[i] = 0;
	dis[s] = 0, vis[s] = 1; Q.push(s);
	while(!Q.empty()) {
		int u = Q.front(); Q.pop();
		vis[u] = 0;
		for(auto i : e[u]) {
			int v = i.v, w = i.w;
			if(dis[v] > dis[u] + w) {
				dis[v] = dis[u] + w;
				cnt[v] = cnt[u] + 1;
				if(cnt[v] >= n + 1) return 0;
				if(!vis[v]) Q.push(v), vis[v] = 1;
			}
		}
	} 
	return 1;
}

signed main()
{
	scanf("%d%d", &n, &m);
	ecnt = 0;
	//x_u - x_v <= w
	for(int i = 1; i <= m; i++) {
		int u, v, w; scanf("%d%d%d", &u, &v, &w);
		e[u].push_back((Node){v, w});
	} 
	for(int i = 1; i <= n; i++) Add(0, i, 0);
	if(!SPFA(0)) printf("NO\n");
	else for(int i = 1; i <= n; i++) printf("%d ", dis[i]);
	return 0;
}
\end{lstlisting}

\subsection{欧拉路径}
\begin{lstlisting}
#pragma GCC optimize(2)
#include<bits/stdc++.h>
using namespace std;

const int MAXN = (int)1e5 + 5;
const int MAXM = (int)2e5 + 5;

stack<int> S;
vector<int> G[MAXN];
int del[MAXN], in[MAXN], out[MAXN], n, m;

void DFS(int u) {
	for(int i = del[u]; i < G[u].size(); i = del[u]) del[u] = i + 1, DFS(G[u][i]);
	S.push(u);
}

signed main()
{
	ios::sync_with_stdio(false); cin.tie(0);
	cin >> n >> m;
	for(int i = 1; i <= m; i++) {
		int u, v; cin >> u >> v;
		G[u].push_back(v); out[u]++; in[v]++;
	}
	for(int i = 1; i <= n; i++) sort(G[i].begin(), G[i].end());
	int s = 1, cnt0 = 0, cnt1 = 0, flag = 1;
	for(int i = 1; i <= n; i++) {
		if(in[i] != out[i]) flag = 0;
		if(out[i] == in[i] + 1) s = i, cnt1++;
		if(in[i] == out[i] + 1) cnt0++;
	}
	if(!flag && !(cnt0 == cnt1 && cnt0 == 1) ) return cout << "No", 0;
	DFS(s);
	while(!S.empty()) cout << S.top() << " ", S.pop();
	return 0;
}
\end{lstlisting}

\subsection{染色法判别二分图}
\begin{lstlisting}
//O(n + m)
int n; // n表示点数
int h[N], e[M], ne[M], idx; // 邻接表存储图
int color[N]; // 表示每个点的颜色,-1表示未染色,0表示白色,1表示黑色
// 参数:u表示当前节点,c表示当前点的颜色
bool dfs(int u, int c) {
	color[u] = c;
	for (int i = h[u]; i != -1; i = ne[i]) {
		int j = e[i];
		if (color[j] == -1) {
			if (!dfs(j, !c)) return false;
		} else if (color[j] == c) return false;
	}
	return true;
}
bool check() {
	memset(color, -1, sizeof color);
	bool flag = true;
	for (int i = 1; i <= n; i ++ ) {
		if (color[i] == -1) if (!dfs(i, 0)) {
			flag = false;
			break;
		}
	}
	return flag;
}
\end{lstlisting}

\subsection{匈牙利算法}
\begin{lstlisting}
bool Match(int x) {
	for(int i = 1; i <= m; i++) {
		if(G[x][i] && !vis[i]) {
			vis[i] = 1;
			if(!a[i] || Match(a[i])) {
				a[i] = x;
				return true;
			}
		}
	}
	return false;
}

int main() {
	scanf("%d%d%d", &n, &m, &e);
	for(int i = 1; i <= e; i++) {
		int u, v; scanf("%d%d", &u, &v);
		G[u][v] = 1;
	}
	int ans = 0;
	for(int i = 1; i <= n; i++) {
		memset(vis, 0, sizeof(vis));
		if(Match(i)) ans++;
	}
	printf("%d", ans);
	return 0;
}
\end{lstlisting}

\subsection{Tarjan}
\begin{lstlisting}
const int N;

int dfn[N], low[N], s[N], vis[N], color[N], top = 0, sum = 0, dep = 0;
int n, m;
vector<int> G[N];

// 求强连通分量 复杂度O(E+V)
void Tarjan(int u) {
	dfn[u] = low[u] = ++dep;
	vis[u] = 1;
	s[++top] = u;
	
	for(int i = 0; i < G[u].size(); i++) {
		int v = G[u][i];
		if(!dfn[v]) Tarjan(v), low[u] = min(low[u], low[v]);
		else if(vis[v]) low[u] = min(low[u], low[v]);
	}
	
	if(dfn[u] == low[u]) {
		color[u] = ++sum;
		vis[u] = 0;
		while(s[top] != u) {
			color[ s[top] ] = sum;
			vis[ s[top] ] = 0;
			top--;
		}
		top--;
	}
}

// 求割点
vector<int> cut; // 存储所有割点
void Tarjan(int u, bool root = true) {
	int tot = 0;
	low[u] = dfn[u] = ++dep;
	for(auto v : G[u]) {
		if(!dfn[v]) {
			Tarjan(v, false);
			low[u] = min(low[u], low[v]);
			tot += (low[v] >= dfn[u]); // 统计满足low[v] >= dfn[u]的子节点数目
		} else low[u] = min(low[u], dfn[v]);
	}
	if (tot > root) // 如果是根,tot需要大于1;否则只需大于0
        cut.push_back(u);
}

// 求割桥
vector<pair<int, int>> bridges;	// 存割桥
void Tarjan(int u) {
	low[u] = dfn[u] = ++dep;
	for(auto v : G[u]) {
		if(!dfn[v]) {
			fa[v] = u;	// 记录父节点
			Tarjan(v);
			low[u] = min(low[u], low[v]);
			if(low[v] > dfn[u]) bridges.emplace_back(u, v);
		} else if (fa[u] != v) // 排除父节点
            low[u] = min(low[u], dfn[v]);
	}
}

void Solve() {
	for(int i = 1; i <= n; i++) if(!dfn[i]) {
		Tarjan(i);
		Tarjan(i, true);
	}
}
\end{lstlisting}

\subsection{Dinic最大流}
\begin{lstlisting}
#include<bits/stdc++.h>
#define int long long

using namespace std;

const int MAXN = 205;
const int MAXM = 5005;

struct Edge {
	int v, w, nxt;
}e[MAXM << 1];

int n, m, s, t, cnt = 1, dep[MAXN], head[MAXN];

void Add(int u, int v, int w) {
	e[++cnt].v = v;
	e[cnt].w = w;
	e[cnt].nxt = head[u];
	head[u] = cnt;
}

bool BFS() {
	for(int i = 1; i <= n + 1; i++) dep[i] = 0;
	dep[s] = 1;
	queue<int> Q;
	Q.push(s);
	while(!Q.empty()) {
		int u = Q.front(); Q.pop();
		for(int i = head[u]; i != -1; i = e[i].nxt) {
			int v = e[i].v;
			if(dep[v] == 0 && e[i].w > 0) {
				dep[v] = dep[u] + 1;
				Q.push(v);
			}
		}
	}
	return (bool)dep[t];
}

int DFS(int u, int in) {
	if(u == t) return in;
	int out = 0;
	for(int i = head[u]; i != -1 && in > 0; i = e[i].nxt) {
		int v = e[i].v;
		if(dep[v] == dep[u] + 1 && e[i].w > 0) {
			int res = DFS(v, min(e[i].w, in));
			e[i].w -= res;
			e[i ^ 1].w += res;	// 反向边(残量网络)
			in -= res;
			out += res; 
		}
	}
	if(out == 0) dep[u] = 0;
	return out;
}

signed main()
{
	memset(head, -1, sizeof(head));
	scanf("%lld%lld%lld%lld", &n, &m, &s, &t);
	for(int i = 1; i <= m; i++) {
		int u, v, w; scanf("%lld%lld%lld", &u, &v, &w);
		Add(u, v, w); Add(v, u, 0);
	}
	int ans = 0;
	while(BFS()) { ans += DFS(s, 1e18); }
	printf("%lld", ans);
	return 0;
}
\end{lstlisting}

\subsection{KM}
\begin{lstlisting}
#include<bits/stdc++.h>
#define int long long

using namespace std;

const int MAXN = 505;
const int INF = (int)1e16;

int n, m;
int G[MAXN][MAXN];
int lmatch[MAXN], rmatch[MAXN];
int pre[MAXN];
int lexpect[MAXN], rexpect[MAXN];	
int lvis[MAXN], rvis[MAXN];
int slack[MAXN];
queue<int> Q;

void aug(int v) {
	int temp;
	while(v) {
		temp = lmatch[ pre[v] ];
		lmatch[ pre[v] ] = v;
		rmatch[v] = pre[v];
		v = temp;
	}
}

void BFS(int s) {
	for(int i = 1; i <= n; i++) lvis[i] = rvis[i] = 0, slack[i] = INF;

	while(!Q.empty()) Q.pop(); 
	Q.push(s);
	
	while(1) {
		while(!Q.empty()) {
			int u = Q.front(); Q.pop();
			lvis[u] = 1;
			for(int v = 1; v <= n; v++) {
				if(!rvis[v]) {
					int gap = lexpect[u] + rexpect[v] - G[u][v];
					if(slack[v] > gap) {
						slack[v] = gap;
						pre[v] = u;
						if(slack[v] == 0) {
							rvis[v] = 0;
							if(!rmatch[v]) { aug(v); return ; } 
							else Q.push(rmatch[v]);
						} 
					}
				}
			}
		}
		
		int d = INF;
		for(int i = 1; i <= n; i++) 
			if(!rvis[i]) d = min(d, slack[i]);
			
		for(int i = 1; i <= n; i++) {
			if(lvis[i]) lexpect[i] -= d;
			
			if(rvis[i]) rexpect[i] += d;
			else slack[i] -= d;
		}
		
		for(int i = 1; i <= n; i++) {
			if(!rvis[i]) {
				if(slack[i] == 0) {
					rvis[i] = 1;
					if(!rmatch[i]) { aug(i); return ; } 
					else Q.push(rmatch[i]);
				}
			}
		}
	}
}

int KM() {
	for(int i = 1; i <= n; i++) lmatch[i] = rmatch[i] = lexpect[i] = rexpect[i] = 0;

	for(int i = 1; i <= n; i++) {
		lexpect[i] = G[i][1];
		for(int j = 2; j <= n; j++) lexpect[i] = max(lexpect[i], G[i][j]);
	}
	
	for(int i = 1; i <= n; i++) BFS(i);
	
	int res = 0;
	for(int i = 1; i <= n; i++) 
		if(rmatch[i]) res += G[rmatch[i]][i];
		
	return res;
}

signed main()
{
	scanf("%lld%lld", &n, &m); 
	for(int i = 1; i <= n; i++) for(int j = 1; j <= n; j++) G[i][j] = -INF;
	for(int i = 1; i <= m; i++) {
		int u, v, w; scanf("%lld%lld%lld", &u, &v, &w);
		G[u][v] = w;
	}
	printf("%lld\n", KM());
	for(int i = 1; i <= n; i++) printf("%lld ", rmatch[i]);
	return 0;
}
\end{lstlisting}

\subsection{轻重链剖分}
\begin{lstlisting}
#include<bits/stdc++.h>
#define int long long

using namespace std;

const int MAXN = (int)1e5 + 5;

struct Node {
	int v, nxt;
}e[MAXN << 1];

int sum[MAXN << 2], tag[MAXN << 2];
int n, m, r, MOD, cnt, e_cnt, v[MAXN];
int head[MAXN], vis[MAXN], fa[MAXN], sze[MAXN], son[MAXN], dep[MAXN];
int idx[MAXN], a[MAXN], top[MAXN];

int ls(int x) { return x << 1; }
int rs(int x) { return x << 1 | 1; }

void F(int l, int r, int p, int k) {
	tag[p] += k; tag[p] %= MOD;
	sum[p] += (r - l + 1) * k; sum[p] %= MOD;
}

void PushUp(int p) {
	sum[p] = sum[ls(p)] + sum[rs(p)];
	sum[p] %= MOD;
}

void PushDown(int l, int r, int p) {
	int mid = (l + r) >> 1;
	F(l, mid, ls(p), tag[p]);
	F(mid + 1, r, rs(p), tag[p]);
	PushUp(p);
	tag[p] = 0;
}

void Build(int l, int r, int p) {
	if(l == r) {
		sum[p] = a[l];
		tag[p] = 0;
		return ;
	}
	int mid = (l + r) >> 1;
	Build(l, mid, ls(p));
	Build(mid + 1, r, rs(p));
	PushUp(p);
}

void SegmentTreeModify(int nl, int nr, int l, int r, int p, int k) {
	if(nl <= l && nr >= r) {
		tag[p] += k; tag[p] %= MOD;
		sum[p] += (r - l + 1) * k; sum[p] %= MOD;
		return ;
	}
	PushDown(l, r, p);
	int mid = (l + r) >> 1;
	if(nl <= mid) SegmentTreeModify(nl, nr, l, mid, ls(p), k);
	if(nr > mid) SegmentTreeModify(nl, nr, mid + 1, r, rs(p), k);
	PushUp(p);
	return ;
}

int SegmentTreeQuery(int nl, int nr, int l, int r, int p) {
	if(nl <= l && nr >= r) {
		return sum[p];
	}
	PushDown(l, r, p);
	int res = 0;
	int mid = (l + r) >> 1;
	if(nl <= mid) res += SegmentTreeQuery(nl, nr, l, mid, ls(p)), res %= MOD; 
	if(nr > mid) res += SegmentTreeQuery(nl, nr, mid + 1, r, rs(p)), res %= MOD;
	return res;
}

void Add(int u, int v) {
	e[++e_cnt].v = v;
	e[e_cnt].nxt = head[u];
	head[u] = e_cnt;
}

void DFS1(int u, int ftr) {
	fa[u] = ftr;
	sze[u] = 1;
	dep[u] = dep[ftr] + 1;
	vis[u] = 1;
	int maxsize = -1;
	for(int i = head[u]; i != -1; i = e[i].nxt) {
		int v = e[i].v;
		if(vis[v]) continue;
		DFS1(v, u);
		sze[u] += sze[v];
		if(sze[v] > maxsize) {
			son[u] = v;
			maxsize = sze[v];
		}
	}
}

void DFS2(int u, int top_u) {
	idx[u] = ++cnt;
	a[cnt] = v[u];
	top[u] = top_u;
	if(son[u] == 0) return ;
	DFS2(son[u], top_u);
	for(int i = head[u]; i != -1; i = e[i].nxt) {
		int v = e[i].v;
		if(v == fa[u] || v == son[u]) continue;
		DFS2(v, v);
	}
}

void Modify(int u, int v, int w) {
	while(top[u] != top[v]) {
		if(dep[top[u]] < dep[top[v]]) swap(u, v);
		SegmentTreeModify(idx[top[u]], idx[u], 1, n, 1, w);
		u = fa[top[u]];
	}
	if(idx[u] > idx[v]) swap(u, v);
	SegmentTreeModify(idx[u], idx[v], 1, n, 1, w);
}

int Query(int u, int v) {
	int res = 0;
	while(top[u] != top[v]) {
		if(dep[top[u]] < dep[top[v]]) swap(u, v);
		res += SegmentTreeQuery(idx[top[u]], idx[u], 1, n, 1);
		res %= MOD;
		u = fa[top[u]];
	}
	if(idx[u] > idx[v]) swap(u, v);
	res += SegmentTreeQuery(idx[u], idx[v], 1, n, 1);
	return res % MOD;
}

signed main()
{
	scanf("%lld%lld%lld%lld", &n, &m, &r, &MOD);
	cnt = 0, e_cnt = 0; memset(head, -1, sizeof(head));
	for(int i = 1; i <= n; i++) scanf("%lld", &v[i]);
	for(int i = 1; i < n; i++) {
		int u, v; scanf("%lld%lld", &u, &v);
		Add(u, v); Add(v, u);
	}
	DFS1(r, 0); DFS2(r, r); Build(1, n, 1);
	while(m--) {
		int opt, x, y, z; scanf("%lld", &opt);
		if(opt == 1) {
			scanf("%lld%lld%lld", &x, &y, &z);
			Modify(x, y, z);	
		} else if(opt == 2) {
			scanf("%lld%lld", &x, &y);
			printf("%lld\n", Query(x, y));
		} else if(opt == 3) {
			scanf("%lld%lld", &x, &z);
			SegmentTreeModify(idx[x], idx[x] + sze[x] - 1, 1, n, 1, z);
		} else {
			scanf("%lld", &x);
			printf("%lld\n", SegmentTreeQuery(idx[x], idx[x] + sze[x] - 1, 1, n, 1) % MOD);
		}
	}
	return 0;
}

/*
input:
5 5 2 24
7 3 7 8 0 
1 2
1 5
3 1
4 1
3 4 2
3 2 2
4 5
1 5 1 3
2 1 3

output: 2 21
*/
\end{lstlisting}

\subsection{树上启发式合并}
\begin{lstlisting}
// 轻重链剖分
void DFS(int u, int fa) {
	dfn[u] = ++tot, node[tot] = u, sze[u] = 1;
	for(auto v : G[u]) {
		if(v == fa) continue;
		DFS(v, u);
		if(sze[v] > sze[ son[u] ]) son[u] = v;
		sze[u] += sze[v];
	}
}

void add(int pos) { }
void del(int pos) { }
int getAns() { }

void DSU(int u, int fa, bool st) {
	for(auto v : G[u]) if(v != fa && v != son[u]) DSU(v, u, 0);
	if(son[u]) DSU(son[u], u, 1);
	for(auto v : G[u]) {
		if(v == fa || v == son[u]) continue;
		for(int i = dfn[v]; i < dfn[v] + sze[v]; i++) add(node[i]);
	}
	add(u);
	ans[u] = getAns();
	if(!st) for(int i = dfn[u]; i < dfn[u] + sze[u]; i++) del(node[i]);
}

void Solve() {
	DFS(1, 0);
	DSU(1, 0, 0);
}
\end{lstlisting}

\subsection{点分治}
\begin{lstlisting}
#include<bits/stdc++.h>
#define pir make_pair
#define pii pair<int, int>
#define fi first
#define se second
using namespace std;
const int MAXN = 1e4 + 5;
const int MAXV = 1e7 + 5;
const int MOD = 1e9 + 7;

int n, m, ans[MAXN], val[MAXN], sze[MAXN], vis[MAXN], centroid;
int Map[MAXN], Cur[MAXN], MapVal[MAXV];
vector<pii> G[MAXN];

void GetCentroid(int u, int fa, int n) {
	sze[u] = 1;
	int maxx = 0;
	for(auto i : G[u]) {
		int v = i.fi, w = i.se;
		if(v == fa || vis[v]) continue;
		GetCentroid(v, u, n);
		if(centroid != -1) return ;
		maxx = max(maxx, sze[v]);
		sze[u] += sze[v];
	}
	maxx = max(maxx, n - sze[u]);
	if(maxx <= n / 2) centroid = u, sze[fa] = n - sze[u];
}

void Calc(int u, int fa, int len) {
	if(len > 1e7) return ;
	Cur[ ++Cur[0] ] = len;
	for(auto i : G[u]) {
		int v = i.fi, w = i.se;
		if(vis[v] || v == fa) continue;
		Calc(v, u, len + w);
	}
}

void Calc(int u) {
	Map[ Map[0] = 1 ] = 0, MapVal[0] = 1;
	for(auto i : G[u]) {
		int v = i.fi, w = i.se;
		if(vis[v]) continue;
		Cur[0] = 0; Calc(v, u, w);
		for(int j = 1; j <= Cur[0]; j++) 
			for(int k = 1; k <= m; k++) 
				if(val[k] - Cur[j] >= 0) 
					ans[k] |= MapVal[ val[k] - Cur[j] ];
		for(int j = 1; j <= Cur[0]; j++) MapVal[ Cur[j] ] = 1, Map[ ++Map[0] ] = Cur[j];
	}
	for(int i = 1; i <= Map[0]; i++) MapVal[ Map[i] ] = 0;
}

void Solve(int u) {
	vis[u] = 1; Calc(u);
	for(auto i : G[u]) {
		int v = i.fi, w = i.se;
		if(vis[v]) continue;
		centroid = -1; GetCentroid(v, 0, sze[v]);
		Solve(centroid);
	}
}

signed main()
{
	ios::sync_with_stdio(false); cin.tie(0); cout.tie(0);
	cin >> n >> m;
	for(int i = 1; i < n; i++) {
		int u, v, w; cin >> u >> v >> w;
		G[u].push_back( pir(v, w) ), G[v].push_back( pir(u, w) );
	}
	for(int i = 1; i <= m; i++) cin >> val[i];
	centroid = -1;
	GetCentroid(1, 0, n);
	Solve(centroid);
	for(int i = 1; i <= m; i++) cout << (ans[i] ? "AYE" : "NAY") << "\n";
	return 0;
}
\end{lstlisting}

\subsection{虚树}
\begin{lstlisting}
//#pragma GCC optimize(2)
#include<bits/stdc++.h>
#define int long long
#define pir make_pair
#define pii pair<int, int>
#define fi first
#define se second
using namespace std;
const int MAXN = 1e6 + 5;
const int INF = 0x3f3f3f3f;

int n, m, cnt, dfn[MAXN], dep[MAXN], top, s[MAXN], vis[MAXN], lca[MAXN][25];
vector<pii> VG[MAXN], vec;
vector<int> G[MAXN];
unordered_map<int, int> Map;

void DFS(int u, int fa) {
	dfn[u] = ++cnt; dep[u] = dep[fa] + 1;
	lca[u][0] = fa;
	for(int i = 1; i <= 20; i++) lca[u][i] = lca[ lca[u][i - 1] ][i - 1];
	for(auto i : G[u]) if(i != fa) DFS(i, u);
}

int LCA(int u, int v) {
	if(dep[u] > dep[v]) swap(u, v);
	for(int i = 20; i >= 0; i--) if((1 << i) & (dep[v] - dep[u])) v = lca[v][i];
	for(int i = 20; i >= 0; i--) if(lca[u][i] != lca[v][i]) u = lca[u][i], v = lca[v][i];
	return u == v ? u : lca[u][0];
}

int Dist(int u, int v) {

}

void Add(int u, int v) {
	int w = Dist(u, v);
	VG[u].push_back( pir(v, w) );
	VG[v].push_back( pir(u, w) );
	if(!Map[u]) Map[u] = ++cnt;
	if(!Map[v]) Map[v] = ++cnt;
}

void VirtualTreeInsert(int u) {
	if(!top) return void(s[++top] = u);
	int l = LCA(s[top], u);
	while(top > 1 && dep[l] < dep[ s[top - 1] ]) Add(s[top - 1], s[top]), --top;
	if(dep[l] < dep[ s[top] ]) Add(l, s[top]), --top;
	if(!top || s[top] != l) s[++top] = l;
	if(s[top] != u) s[++top] = u;
}

void BuildVirtualTree(vector<pii> vec, int k) {
	sort(vec.begin(), vec.end());
	for(auto i : Map) VG[i.fi].clear();
	top = cnt = 0; Map.clear(); 
	s[++top] = 1, Map[1] = ++cnt;
	for(auto i : vec) VirtualTreeInsert(i.se);
	for(int i = 1; i < top; i++) Add(s[i], s[i + 1]);
}

signed main()
{
	ios::sync_with_stdio(false); cin.tie(0); cout.tie(0);
	cin >> n;
	for(int i = 1; i < n; i++) {
		int u, v; cin >> u >> v;
		G[u].push_back(v), G[v].push_back(u);
	}
	DFS(1, 0); cnt = 0;
	cin >> m;
	while(m--) {
		int k; cin >> k; vec.clear();
		for(int i = 1; i <= k; i++) {
			int cur; cin >> cur;
			vec.push_back( pir(dfn[cur], cur) );
		}
		BuildVirtualTree(vec, k);
	}
	return 0;
}
\end{lstlisting}

\subsection{树哈希}
\begin{lstlisting}
// 复杂度O(nlogn), n为节点数量
#include<bits/stdc++.h>
using namespace std;
const int MAXN = 1e6 + 5;

vector<int> g[MAXN];
unordered_map<int, int> num;
map<vector<int>, int> Hash;
int hash_cnt;

int dfs(int u, int f) {
	vector<int> vec;
	for(auto &v : g[u]) {
		if(v == f) continue;
		vec.push_back( dfs(v, u) );
	} 
	sort(vec.begin(), vec.end() );
	if(Hash[vec] == 0) Hash[vec] = ++hash_cnt;
 	return Hash[vec];
}

void init() {
	hash_cnt = 0;
	Hash.clear();
}

signed main() {
	//ios::sync_with_stdio(false); cin.tie(0); cout.tie(0);
	init();
	int n; cin >> n;
	for(int j = 1; j <= n; j++) {
		int f; cin >> f;
		if(f == 0) continue;
		g[f].push_back(j); g[j].push_back(f);
	}
	
	/*
	树哈希返回对应子树的哈希值,
	如需要比较两棵树的哈希值,可以通过求重心的方式固定根节点;
	若重心有两个,分别固定根节点求哈希值即可。
	*/
	int h = dfs(centroid, 0);
	if(num[h] == 0) num[h] = i;
	cout << num[h] << "\n";
	
	for(int j = 0; j <= n; j++) g[j].clear();
} 
\end{lstlisting}