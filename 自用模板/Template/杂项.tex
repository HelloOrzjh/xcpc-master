\section{杂项}

\subsection{STL}
\begin{lstlisting}
// 万能(误)算法头文件(部分)
#include <algorithm>
using namespace std; 
int main() {
    iterator begin, end; // 指代某种数据结构首尾迭代器
    T i, x, a, b;
    
    sort(begin, end, <cmp>); // 排序函数,默认从小到大
    // 遇到需要特殊排序的需要编写cmp函数 or 重载内部运算符
    next_permutation(begin, end); // 下一个排列
    prev_permutation(begin, end); // 前一个排列
    
    set_union(begin(a), end(a), begin(b), end(b), begin(c));
    // 取两个有序序列a、b的并集,存放到c中
    set_intersection(begin(a), end(a), begin(b), end(b), begin(c));
    // 取两个有序序列a、b的交集,存放到c中
    set_difference(begin(a), end(a), begin(b), end(b), begin(c));
    // 取两个有序序列a、b的差集,存放到c中
    unique(begin, end); // 有序数据去重
    merge(begin(a), end(a), begin(b), end(b), begin(c), cmp);
    // 合并两个有序序列a、b,存放到c中,cmp可定义新序列排列方式
    
    lower_bound(begin, end, x); // 返回x的前驱迭代器
    // 在普通的升序序列中,x的前驱指的是第一个大于等于x的值
    upper_bound(begin, end, x); // 返回x的后继迭代器
    // 在普通的升序序列中,x的后继指的是第一个大于x的值
    // 上述两个函数时间复杂度为O(log2(n)),内部实现是二分
    // 如果找不到这样的值,会返回end
    
    find(begin, end, x); // O(n)查找x
    binary_search(begin, end, x) // 二分查找x,返回bool
    min(a, b); max(a, b); // 返回a、b中的最小/最大值
    fill(begin, end, x); // 往容器的[begin, end)内填充x
    swap(a, b); // 交换a、b的值
    
    return 0;
}

// 动态数组(vector)、双向链表(list)
#include <vector>
#include <list>
using namespace std;
int main() {
    T i;
    unsigned int n, x;
    bool flag;
    iterator it;
    
    // 动态数组部分
    // 注意vector的空间需要预留两倍大小
    vector<T> v;
    v.push_back(i); // 往数组尾添加一个元素i
    v[x]; // 访问第x - 1个元素
    v.begin(); // 返回头元素的迭代器
    v.end(); // 返回末尾迭代器(尾元素的下一个)
    n = v.size(); // 数组中元素数量
    v.pop_back(); // 删除最后一个元素
    v.erase(it);  // 删除某个的元素
    v.insert(x, i);  // 在x位置插入元素i
    // erase、insert时间复杂度为O(n)
    v.clear(); // 清空数组,不释放空间
    flag = v.empty(); // 判断数组是否为空(真值)
    
    // 链表部分
    list<T> li;
    li.push_front(i); // 在链头添加一个元素i
    li.push_back(i); // 在链尾添加一个元素i
    li.pop_front(i); // 删除链表头元素
    li.pop_back(i); // 删除链表尾元素
    li.erase(it);  // 删除某个的元素
    li.insert(x, i);  // 在x位置插入元素i O(n)
    li.begin(); // 返回头元素的迭代器
    li.end(); // 返回末尾迭代器(尾元素的下一个)
    n = li.size(); // 链表中元素数量
    li.remove(i); // 删除链表中所有值为i的元素
    li.unique(); // 移除所有连续相同元素,留下一个
    li.reverse(); // 反转链表
    li.clear(); // 情况链表,不释放空间
    
    return 0;
}

// 普通队列、双端队列、优先队列
#include <queue> // 队列头文件
#include <deque> // 双端队列头文件
using namespace std;
int main() {
    T i; 
    unsigned int n, x; 
    bool flag;
    
    // 普通队列部分,注意queue没有迭代器
    queue<T> q, tmp_q;  // 定义普通队列
    q.push(i); // 队尾插入元素i
    q.pop();   // 弹出队首元素
    i = q.front(); // 访问队首元素
    i = q.back();  // 访问队尾元素
    n = q,size();  // 队内元素数量
    flag = q.empty(); // 判断队列是否为空(真值)
    q.swap(tmp_q); // 交换两个队列元素
    
    // 优先队列部分,注意其没有迭代器
    priority_queue<T> pq; // 定义优先队列
    pq.push(i); // 队尾插入元素i
    pq.pop();   // 弹出队首元素
    i = pq.top(); // 访问队首元素
    n = q,size();  // 队内元素数量
    flag = q.empty(); // 判断队列是否为空(真值)
    q.swap(tmp_q); // 交换两个队列元素
    // 注意优先队列内部是使用<运算符,默认大根堆
    // 可以采用重载运算符或加入运算符类自定义排列方式
    // 例:priority_queue<T, vector<T>, greater<T> > 小根堆
    /*
        struct node {
            int x, y;
        };
        bool operator < (node a, node b) {
            // 这里注意是<右边的元素会放在前面
            if(a.x != b.x) return a.x < b.x;
            else return a.y < b.y;
        }
        priority_queue<node>
    */
    
    // 双端队列部分
    // 注意deque用到了map来映射,时间复杂度上常数略大
    deque<T> dq; // 定义双端队列
    // 可以称为vector、list、queue的结合体
    // 用法类似,这里只给代码不做注释
    dq.push_back(i);
    dq.push_front(i);
    dq.front();
    dq.back();
    dq.pop_front();
    dq.pop_back();
    dq.begin();
    dq.end();
    dq[x];
    n = dq.size();
    flag = dq.empty();
    dq.insert(x, i);
    
    return 0;
}

// 栈
#include <stack>
using namespace std;
int main() {
    T i;
    unsigned int n;
    bool flag;
    
    stack<T> st; // 注意stack没有迭代器
    st.push(i); // 往栈顶加入一个元素
    st.pop(); // 弹出栈顶元素
    i = st.top(); // 获得栈顶元素的值
    flag = st.empty(); // 判断是否为空(真值)
    n = st.size(); // 获得栈内元素个数
    
    return 0;
}

// pair(成组)、set(有序元素序列)
#include <set>
#include <pair>
using namespace std;
int main() {
    T i;
    T1 t1;
    T2 t2;
    iterator it;
    unsigned int n;
    bool flag;
    
    // pair是将两种元素组成一对
    pair<T1, T2> p;
    p = make_pair(t1, t2); // 将t1、t2构造成一对
    // pair支持比较,遵循字典序
    p.first;  // 访问第一个元素,这里是t1
    p.second; // 访问第二个元素,这里是t2
    
    // set内部是RB-tree维护
    set<int> st; // 注意,set内元素不重复
    st.insert(i); // 往set内插入一个元素i
    	// 时间复杂度O(log2(n)) 这里会返回一个<pair>迭代器
    	// first指向插入元素后所在的迭代器
    	// second指向是否插入成功(真值)
    st.begin(); // 返回首迭代器
    st.end(); // 范围尾迭代器
    st.erase(it); st.erase(i);
    	// 删除某个元素
    st.equal_range(i); // 返回几何中与i相等的上下限两个迭代器
    flag = st.empty(); 
    n = st.size();
    st.clear();
    // set内置了lower_bound和upeer_bound函数
    // 用法和algorithm的一样
    
    // 可重复元素set
    multiset<int> mst;
    // 用法与set大致相同
    // 唯一不同只在删除函数上
    mst.erase(i); // 会删除所有值为i的元素
    
    return 0;
}

// 如果需要给set自定义排序顺序
struct CMP {
    bool operator() (const int& a, const int& b) const {
        return a > b; // 返回真值则代表左边的值优先级高
    }
};
multiset<int, CMP> mst;

// map(映射)
#include <map>
using namespace std;
int main() {
    T1 t1;
    T2 t2;
    
    // map将两种元素做映射,一种指向另一种
    // 内部也是RB-tree维护
    map<T1, T2> mp;
    mp[t1] = t2; // 直接让t1对应到t2
   	mp[t1]; // 访问t1对应的内容,时间复杂度O(log2(n))
    // 如果t1没有指向任何内容,则会返回T2类型的初始值
    
    return 0;
}

// 一些C++的功能/特性
#include <bits/stdc++.h> // 标准库头文件
using namespace std;
int main() {
    
    __int128 a; // 128位整数,最大值大概10^38次方
   	// C++11以上可用,无法用标准方法读入
    
    cin.tie(0); cout.tie(0);
    ios::sync_with_stdio(false); 
    // 关闭cin、cout同步流,此举后不可混用scanf/printf
    
    auto x; // 自动变量,可以是任意属性
    // 举个例子
    std::set<int> st;
    std::for(auto i:st); // C++版for_each
    // 其中i是auto变量,也可改成set<int>::iterator
    
    // 所有的STL容器push/insert操作都可替换为emplcace
    // 速度上减小常数(不用临时变量)
    // 例:
    int i;
    std::set<int> st;
    st.emplace(i);
    std::vector<int> vc;
    vc.emplace_back(i);
    
    return 0;
}

// 强大的pb_ds库
#include <bits/stdc++.h>
#include <bits/extc++.h> // 扩展库头文件
/*
 * 这里如果没有bits/extc++.h的话需要
 * ext/pb_ds/tree_policy.hpp
 * ext/pb_ds/assoc_container.hpp
 * ext/pb_ds/priority_queue_policy.hpp
 * ext/pb_ds/trie_policy.hpp
 * ext/rope
 * ...
 */
using namespace __gnu_pbds;
using namespace __gnu_cxx;
int main() {

    // 哈希表部分,用法与map一样,效率在C++11以下效率高
    // 注意,这部分在namespace __gnu_pbds下
    cc_hash_table<string, int> mp1; // 拉链法
    gp_hash_table<string, int> mp2; // 查探法(快一些)

    // 优先队列部分,比STL中高级
    priority_queue<int, std::greater<int>, TAG> pq;
    /*
     * 第一个参数是数据类型
     * 第二个是排序方式
     * 第三个是堆的类型
     * 其中堆的类型有下面几种
     * pairing_heap_tag
     * thin_heap_tag
     * binomial_heap_tag
     * c_binomial_heap_tag
     * binary_heap_tag
     * 其中pairing_heap_tag最快
     * 并且这个东西是带默认参数的,只需要定义一个int
     */
    // 比STL中的优先队列多了join和迭代器
    // 例子:
    priority_queue<int> pq1, pq2;
    pq1.join(pq2); // 会将pq2合并到pq1上
    pq1.begin(); pq1.end(); // 可遍历

    // 红黑树(平衡树)部分,与set相似,但更快
    tree <
        int,
        null_type,
        std::less<>,
        rb_tree_tag,
        tree_order_statistics_node_update
    > t, tre;
    /*
     * int 关键字类型
     * null_type 无映射(低版本g++为null_mapped_type)
     * less<int> 从小到大排序
     * rb_tree_tag 红黑树(splay_tree_tag splay)
     * tree_order_statistics_node_update 结点更新
     */
    int i, k;
    t.insert(i); // 插入
    t.erase(i); // 删除
    t.order_of_key(i);
    // 询问这个tree中有多少个比i小的元素
    t.find_by_order(k);
    // 找第k + 1小的元素的迭代器,如果order太大会返回end()
    t.join(tre); // tre合并到t上
    t.split(i, tre); // 小于等于i的保留,其余的属于tre
    // 基本操作有size()/empty()/begin()/end()等
    // 同样内置lower_bound/upper_bound

    // 可持久化平衡树部分
    // 注意,这部分在namespace __gun_cxx下
    rope<char> str;
    // 待我学习完后再更新

    return 0;
}
\end{lstlisting}

\subsection{优化}
\begin{lstlisting}
// 关闭iostream同步流
std::ios::sync_with_stdio(false); std::cin.tie(0);
// 如果编译开启了 C++11 或更高版本,建议使用 std::cin.tie(nullptr);
// 注意,此后不可和scanf/printf混用

// 普通快读快写
inline void read_int(int &X) {
    X = 0; int w = 0; char ch = 0;
    while(!isdigit(ch)) w |= ch=='-', ch = getchar();
    while( isdigit(ch)) X = (X<<3)+ (X<<1) + (ch-48), ch = getchar();
    X = w ? -X : X;
}

inline void write_int(int x) {
    static int sta[65];
    int top = 0;
    do {
        sta[top++] = x % 10, x /= 10;
    } while(x);
    while(top) putchar(sta[--top] + 48); 
}

// fread快读
namespace fastIO {
#define BUF_SIZE 100000
    //fread -> read
    bool IOerror = 0;
    inline char nc() {
        static char buf[BUF_SIZE], *p1 = buf + BUF_SIZE, *pend = buf + BUF_SIZE;
        if (p1 == pend) {
            p1 = buf;
            pend = buf + fread(buf, 1, BUF_SIZE, stdin);
            if (pend == p1) {
                IOerror = 1;
                return -1;
            }
        }
        return *p1++;
    }
    inline bool blank(char ch) {
        return ch == ' ' || ch == '\n' || ch == '\r' || ch == '\t';
    }
    inline void read(int &x) {
        char ch;
        while (blank(ch = nc()));
        if (IOerror) return;
        for (x = ch - '0'; (ch = nc()) >= '0' && ch <= '9'; x = x * 10 + ch - '0');
    }
#undef BUF_SIZE
};
using namespace fastIO;

// 手动扩栈
#pragma comment(linker, "/STACK:1024000000,1024000000") 

// O2 O3优化
#pragma GCC optimize(2)
#pragma GCC optimize(3)
\end{lstlisting}

\subsection{对拍}
\begin{lstlisting}
@echo off
:loop
随机数据生成.exe
暴力.exe
正解.exe
fc 暴力.out 正解.out
if not errorlevel 1 goto loop
pause
:end
\end{lstlisting}

\subsection{Java大整数类}
\begin{lstlisting}
// Java大整数
import java.util.*;
import java.math.*;

public class BigInt {
    static Scanner in = new Scanner(System.in); // 定义输入对象
    public static void main(String[] args) {
        BigInteger bigInt_1 = new BigInteger("100");
        BigInteger bigInt_2 = BigInteger.valueOf(123);
        //两种定义方式,建议使用第一种
        bigInt_1.add(bigInt_2); // 加法
        bigInt_1.subtract(bigInt_2); // 减法
        bigInt_1.multiply(bigInt_2); // 乘法
        bigInt_1.divide(bigInt_2);// 除法,向下取整
        bigInt_1.divideAndRemainder(bigInt_2);
            // 返回一个BigInteger[],包含商和余数
        bigInt_1.remainder(bigInt_2); // 取余数,与this同符号
        bigInt_1.mod(bigInt_2); // 取模,模数只能为正整数
        bigInt_1.pow(10); // a^b
        bigInt_1.gcd(bigInt_2); // 最大公约数
        bigInt_1.compareTo(bigInt_2);
            // 比较大小,<0表示this小,0表示相等,>0表示this大
        bigInt_1.equals(bigInt_2); // 真值为相等
        bigInt_1.negate(); // -a
        bigInt_1.abs(); // 绝对值
        bigInt_1.min(bigInt_2); // 最小值
        bigInt_1.max(bigInt_2); // 最大值
        BigInteger a = in.nextBigInteger(); // 读入
        System.out.println(bigInt_1); // 输出
    }
}

// Java大实数
import java.util.*;
import java.math.*;

public class Big {
    static Scanner in = new Scanner(System.in); // 定义输入对象
    public static void main(String[] args) {
        BigDecimal bigDec_1 = new BigDecimal("123.1");
        BigDecimal bigDec_2 = BigDecimal.valueOf(233.213);
        // 唯有除法与BigInteger不同,无限小数需要规定保留位数
        int scale = 3; // 保留位数
        bigDec_1.divide(bigDec_2, scale, RoundingMode.HALF_UP);
        /*
            第三个参数为保留方式,有以下几种:
            CEILING 正无穷大方向取整
            FLOOR 负无穷大方向取整
            DOWN 向 0 的方向取整
            UP	正数向正无穷大取整,负数向负无穷大取整
            HALF_UP 常用的4舍5入
            HALF_DOWN 6,7,8,9 向上取整
            HALF_EVEN 小数位是5时,判断整数部分是奇数就进位,6,7,8,9 进位
         */
    }
}
\end{lstlisting}